\documentclass[twocolumn,superscriptaddress,prb,10pt]{revtex4-1}
%\usepackage{verbatim}
\usepackage{amsmath,amssymb}
%\usepackage{unicode-math}
\usepackage{graphicx}
\usepackage{color}
\usepackage[colorlinks,bookmarks=false,citecolor=blue,linkcolor=red,urlcolor=blue]{hyperref}
\usepackage{times}





%\usepackage[dvips]{graphics}



%%%%%%%%%%%%%   useful shortcuts %%%%%%%%%%%%%%%%%%%%%%%%%%%%%%%%%

\def \h{\hbar}   %  \h won't be used for any greek letter
\def \refe{\eqref}
\def \trm{\textrm}
\def \f{\frac}
\def \hf{\tfrac{1}{2}}    \def \HF{\dfrac{1}{2}}
\def \u{\uparrow}
\def \d{\downarrow}

\def \ord{\mathcal{O}}
\newcommand{\ra}{\rightarrow}   \newcommand{\lra}{\longrightarrow}  

\def\lba{\left(}    \def\rba{\right)}
\def\lbc{\left[}    \def\rbc{\right]}
\def\lbb{\left\{}    \def\rbb{\right\}}

\def\tr{\textrm{Tr}}
\def\refe{\eqref}

\newcommand{\bra}[1]{\langle\left.{#1}\right|}
\newcommand{\ket}[1]{\left|{#1}\right.\rangle}
\newcommand{\xpct}[1]{\langle{#1}\rangle}    % expectatn value

%\DeclareMathOperator{\tr}{tr}

%%%%%%%%%%%%%%%%%%%%%%%%%%%%%%%%%%%%%%%%%%%%%%%%%%%%%%%

\newcommand{\vp}{{\bf p}}  % usual vector quantities
\newcommand{\vq}{{\bf q}}  % double bracketing not required with \vec
\newcommand{\vk}{{\bf k}}  % but required with \bf

\renewcommand{\vr}{{\bf r}} 
\newcommand{\vx}{{\bf x}}

\newcommand{\hc}{\hat{c}}  \newcommand{\hcd}{\hat{c}^\dag} 
\newcommand{\hd}{\hat{d}}  \newcommand{\hdd}{\hat{d}^\dag} 






\begin{document}

\title{Numerical construction of the GGE in integrable models: A Hilbert space Monte Carlo 
approach} 

\author{Vincenzo Alba}
\affiliation{International School for Advanced Studies (SISSA),
Via Bonomea 265, 34136, Trieste, Italy, 
INFN, Sezione di Trieste}

\author{Maurizio Fagotti}
\affiliation{
D\'epartement de Physique, Ecole normale superieure, CNRS, 24 rue Lhomond, 75005
Paris, France}

\date{\today}




\begin{abstract} 


\end{abstract}

% \pacs{73.43.Cd, 71.10.Pm  {\tt check!}}

\maketitle


%#############-- INTRODUCTION --########################################
\paragraph*{Introduction.---}

The issue of how statistical ensembles arise from the out-of-equilibrium 
dynamics in {\it isolated} quantum many-body system is still a fundamental, 
yet challenging, problem. The main motivation of the renewed interest 
in this topic is the high degree of control reached in out-of-equilibrium 
experiments with cold atomic gases~\cite{greiner-2002,kinoshita-2006,
hofferberth-2007,bloch-2008,trotzky-2012,gring-2012,cheneau-2012,
schneider-2012,kunhert-2013,langen-2013,meinert-2013,fukuhara-2013,
ronzheimer-2013,braun-2014}. 
The paradigm experiment is the so-called gloabal {\it quantum quench}, in 
which a system is initially prepared in an eigenstate $|\Psi_0\rangle$ of a 
many-body Hamiltonian ${\mathcal H}$. Then a global parameter of ${\mathcal H}$ 
is suddenly changed, and the system is let to evolve unitarily under the 
new Hamiltonian ${\mathcal H}'$. 
At long times after the quench the system is expected to equilibrate, due to 
dephasing~\cite{barthel-2008}, as confirmed by experiments. 
On the other hand, in integrable models the 
presence of non-trivial {\it local} conserved quantities, besides the energy, 
strongly affects the dynamics and the nature of the steady state. As for now,  
it is still unclear wheather the steady-state can be described by a statistical 
ensemble, and how to construct it. 

It has been proposed that the steady-state expectation value of a generic local 
operator $\hat{\mathcal O}$ is described by a Generalized Gibbs Ensemble~\cite{
rigol-2008} (GGE) as $\langle\hat{\mathcal O}\rangle\equiv\textrm{Tr}({\mathcal O}
\rho^{GGE})$. The GGE density matrix $\rho^{GGE}$ extends the Gibbs density matrix 
by including  all the local conserved quantities $\hat {\mathcal I}_j$ (charges) as 
%
\begin{equation}
\rho^{GGE}=Z^{-1}\exp\big(-\lambda_j\hat{\mathcal I}_j\big). 
\label{rho-gge}
\end{equation}
%
Here we assume summation over the repeated index $j$, $Z$ is a normalization 
factor, and $\lambda_j$ are Lagrange multipliers to be fixed by imposing 
$\langle\Psi_0|\hat{\mathcal I}_j|\Psi_0\rangle=\langle\hat{\mathcal I}_j\rangle$. 
In~\eqref{rho-gge} $\hat{\mathcal I}_2$ denotes the Hamiltonian. 
In realistic situations one deals with the truncated GGE (TGGE), i.e., 
considering only a finite subset of the charges. While the validity of the GGE 
has been largely confirmed in non-interacting field theories, in interacting 
ones the scenario is not clear. For Bethe ansatz solvable models the so-called 
quench-action method~\cite{caux-2013} allows to characterize the steady state 
analytically, provided that the overlap between $|\Psi_0\rangle$ and the 
eigenstates of ${\mathcal H}'$ are known. Remarkably, in several cases the 
quench-action is in disagreement with the TGGE, whereas it seems to be 
supported by numerical simulations~\cite{pozsgay-2014}. 
While one might argue that the GGE can be repaired by including more charges, 
no quantitative study has been conducted yet. One intriguing possibility is 
that local charges are not enough and one has to include quasi-local 
ones~\cite{ilievski-2015}. 

On the numerical side, numerical methods, such as the time dependent density 
matrix renormalization group~\cite{white-2004,daley-2004} (tDMRG), have been mostly 
used to simulate the dynamics in microscopic models. However, no numerical attempt 
of exploring the GGE itself has been undertaken yet. The aim of this paper is to 
provide a Monte-Carlo-based framework for studying  the GGE and its possible extensions. 
The method is deviced for finite-size Bethe ansatz solvable models. Thermodynamic 
quantities can be obtained by standard finite-size scaling analysis.  
The method relies on the detailed knowledge of the Hilbert space of Bethe ansatz 
solvable models, and, in particular, in the so-called Bethe-Takahashi (B-T) equations. 
The key idea is to sample the model Hilbert space according to the GGE probability 
measure~\eqref{rho-gge}. Notice that for the Gibbs ensemble a similar approach has 
been proposed in Ref.~\onlinecite{gu-2005}.
The method allows to construct the GGE expectation value of a generic (both local 
and non-local) observable $\hat {\mathcal O}$, provided that its expression in terms 
of the roots of the B-T equations (rapidities) is known. Remarkably, we show that is 
also possible to extract the so-called rapidity distributions, which encode the 
complete information about the GGE steady state in the thermodynamic limit. 
Finally, we should mention that the GGE expectaion values of local observables 
could be computed using exact diagonalization or Quantum Monte Carlo. However, 
both these methods require the operatorial expression of the conserved charges, 
while the approach presented here relies only on their expression (typically much 
simpler) in terms of the B-T roots. Moreover, the Monte Carlo approach allows  
to extended the GGE including arbitrary functions of the B-T roots in a trivial 
manner, which would be useful in trying to incorporate the quasi-local charges.   

To benchmark the method here we focus on the spin-$\frac{1}{2}$ isotropic 
Heisenberg chain ($XXX$ chain) that is the venerable prototype of integrable 
models. We restrict ourselves to the TGGE constructed including the first 
three conserved charges $\hat{\mathcal I}_2,\hat{\mathcal I}_3,\hat{\mathcal I}_4$, 
varying the associated Lagrange multipliers $\lambda_j$. 
We numerically investigate the GGE expectation values of the charges $\langle\hat
{\mathcal I}_j\rangle$ and the variance of their ensemble fluctuations 
$\sigma^2(\hat{\mathcal I}_j)$. These are related to well-known physical observables 
such as the energy density, the energy current, the specific heat, and the energy 
Drude weight. We also consider the average magnetization and the spin susceptibility. 
Interestingly, finite-size corrections decay expontially with the chain size, and 
results for moderately large chains are indistinguishable from the thermodynamic 
limit. Moreover, the Monte Carlo data perfectly agree with the analytical 
results obtained using the Generalized Thermodynamic Bethe Ansatz (GTBA) approach. 
This result provides the first direct numerical check of the GTBA for the $XXX$ 
chain. Finally, we extract the first few rapidity distributions corresponding to the 
the Gibbs ensemble steady state at several temperatures, and the GGE. In both 
cases finite-size effects are small, especially for small rapidities, which reflect 
long-wavelength physics. For the Gibbs ensemble we compare with standard finite 
temperature Thermodynamic Bethe Ansatz (TBA) results, finding perfect agreement. 






%#############-- THE HEISENBERG SPIN CHAIN --########################################
\paragraph*{The Heisenberg spin chain.---}

The isotropic spin-$\frac{1}{2}$ Heisenberg ($XXX$) chain is defined by the 
Hamiltonian 
%
\begin{align}
\label{xxx-ham}
{\mathcal H}\equiv J\sum\limits_{i=1}^L\left[\frac{1}{2}(S_i^+S^-_{i+1} 
+S_i^{-}S_{i+1}^+)+S_i^zS_{i+1}^z\right],  
\end{align}
%
where $S^{\pm}_i\equiv (\sigma_i^x\pm i\sigma_i^y)/2$ are spin operators 
acting on the site $i$ of the chain, $S_i^z\equiv\sigma_i^z/2$, and 
$\sigma^{x,y,z}_i$ the Pauli matrices. We fix $J=1$ and use periodic 
boundary conditions identifying sites $L+1$ and $1$. The total magnetization 
$S_{T}^z\equiv\sum_iS_i^z=L/2-M$, with $M$ number of down spins (particles), 
commute with~\eqref{xxx-ham}. Thus, the eigenstates of~\eqref{xxx-ham} can 
be labelled by $M$. 

%%%%%%%%%%%%%%%%%%%%%%%%%%%%%%%%%e
\begin{figure*}[t]
\includegraphics*[width=0.93\linewidth]{./draft_figs/fig1}
\caption{The Generalized Gibbs Ensemble (GGE) for the Heisenberg spin chain with 
 $L=16$ sites: numerical results obtained using the Hilbert space Monte 
 Carlo sampling. Only the first three conserved charges ${\mathcal I}_n$ ($n=1,2,3$), 
 with associated Lagrange multipliers $\lambda_n$, are included in the GGE. Here 
 ${\mathcal I}_2$ is the Hamiltonian and $\lambda_2\equiv\beta$ the inverse 
 temperature. In all the panels different symbols correspond to different  values 
 of $\lambda_3,\lambda_4$. The circles correspond to the Gibbs ensemble, i.e., 
 $\lambda_3=\lambda_4=0$. (a) The GGE average $\langle {\mathcal I}_2/L\rangle$ 
 plotted as a function of $\beta$. (b) Variance of the GGE fluctuations $\sigma^2(
 {\mathcal I}_2/L)\equiv \langle ({\mathcal I}_2/L)^2\rangle-\langle {\mathcal I}_2/L
 \rangle^2$ as a function of $\beta$. (c)(d) and (e)(f): Same as in (a)(b) for 
 ${\mathcal I}_3$ and ${\mathcal I}_4$, respectively. In all panels the dash-dotted 
 lines  are the analytical results obtained using the Generalized Thermodynamic Bethe 
 Ansatz (GTBA). (g) The GGE expectation value of the total magnetization 
 $\langle S_z\rangle$. Notice that $\langle S_z\rangle=0$ due to the $SU(2)$ 
 invariance of the conserved charges. (h) $\chi/\beta$ plotted versus $\beta$, with 
 $\chi$ being the magnetic susceptibility per site. 
}
\label{fig1}
\end{figure*}
%%%%%%%%%%%%%%%%%%%%%%%%%%%%%%%%%%

In the Bethe ansatz framework each eigenstate of~\eqref{xxx-ham} is univocally 
identified by a set of $M$ complex parameters (so-called rapidities) 
$\{x_\alpha\in\mathbb{C}\}_{\alpha=1}^M$. In the thermodynamic limit 
$L\to\infty$ the rapidities ${x_\alpha}$ form ``string'' patterns along 
the imaginary direction in the complex plane (string hypothesis). The 
rapidities forming a string of length $1\le n\le M$ (so-called $n$-string) 
are parametrized as $x_{n;\gamma}^{(j)}=x_{n;\gamma}-i(n-1-2j),\quad j=0,1,
\dots,n-1,$, where $x_{n;\gamma}\in\mathbb{R}$ is the real part of the 
string (string center), $j$ labels the different rapidities in the same 
$n$-string, and$\gamma$ denotes strings of the same length but with 
different centers. Although the string hypothesis is not correct for finite 
chains, deviations typically, i.e., for most of the eigenstates, decay 
exponentially with system size. Physically, the $n$-strings correspond 
to eigenstate components containing bound states of $n$ particles. 
The string centers $x_{n;\gamma}$ are solutions of the Bethe-Takahashi equations 
%
\begin{equation}
L\vartheta_n(x_{n;\gamma})=2\pi I_{n;\gamma}+\sum\limits_{(m,\beta)
\ne(n,\gamma)}\Theta_{m,n}(x_{n;\gamma}-x_{m;\beta}).
\label{bt-eq}
\end{equation}
%
Here $\vartheta_n(x)\equiv2\arctan(x/n)$, $\Theta_{m,n}(x)$ is the scattering 
phase between different rapidities, and $I_{n;\gamma}\in\frac{1}{2}
\mathbb{Z}$ are the so-called Bethe-Takahashi quantum numbers. 
The $I_{n;\gamma}$ obey the upper bound $|I_{n;\gamma}|\le I_{\textrm{MAX}}(n,L,M)$, 
with $I_{\textrm{MAX}}$ a known function of $n,M,L$. 
Every choice of $I_{n;\gamma}$ identifies an eigenstate of~\eqref{xxx-ham}. 
Notice that each eigenstate contains strings of different lengths. We define 
the ``string content'' of an eigenstate as ${\mathcal S}\equiv\{s_1,\dots,
s_M\}$, with $0\le s_n\le \lfloor M/n\rfloor$ the number of $n$-strings. By 
definition one has $\sum_{j}js_j=M$. 

Besides the total magnetization and the momentum, the $XXX$ chain has non-trivial 
{\it local} conserved charges ${\mathcal I}_j$, with $[{\mathcal I}_j,
{\mathcal I}_k]=0\,\forall j,k$. These are obtained as  
%
\begin{equation}
\label{I-def}
\left.{\mathcal I}_{j+1}\equiv\frac{i}{(j-1)!}\frac{d^j}{dy^j}\log(\Lambda
(\{x_{n;\gamma}\},y))
\right|_{y=i}.
\end{equation}
%
Here $\Lambda$ in the Algebraic Bethe ansatz is the eigenvalue of the quantum 
transfer matrix $T(y)$, with $y$ the spectral parameter. The dependence of 
$\Lambda$ on the rapidities $x_{n;\gamma}$ is known. One can check that 
${\mathcal I}_2={\mathcal H}$. The range of ${\mathcal I}_j$ increases 
linearly with $j$, i.e., larger $j$ correspond to less local charges. 
Remarkably, the eigenvalues of ${\mathcal I}_j$ over a generic eigenstate are 
obtained by summing the contributions of the different string sectors independently. 
For instance, the energy is obtained as $E=\sum_n E_n$, where $E_n=2\sum_\gamma n/
(n^2+x^2_{n;\gamma})$. 



%#############-- HILBERT SPACE MONTE CARLO --########################################
\paragraph*{Hilbert space Monte Carlo sampling.---}


For a finite chain the GGE ensemble can be generated by sampling the eigenstates 
of~\eqref{xxx-ham} with the probability~\eqref{rho-gge}. This can be done efficiently 
using Monte Carlo. One starts with an initial $M$ particle eigenstate of~\eqref{xxx-ham}, 
with string content ${\mathcal S}=\{s_1,\dots,s_M\}$, identified by 
Bethe-Takahashi quantum number configuration ${\mathcal C}=\{I_{n;\gamma}\}_{n=1}^M$ 
($\gamma=1,\dots,s_n$). Let us denote the expectation values of the conserved 
charges as $\{{\mathcal I_j}\}$. The basic idea is to generate a new eigenstate with a 
Metropolis update. Specifically, each Monte Carlo step (mcs) consists of three moves:
%
\begin{enumerate}
\item Choose a new particle number sector $M'$ and a string content ${\mathcal S}'$ 
 with probability $P(M',{\mathcal S}')$.
\item Generate a quantum number configuration ${\mathcal C}'$ compatible with 
 the ${\mathcal  S}'$ obtained in step $1$ and solve the Bethe-Takahashi 
 equations~\eqref{bt-eq} to extract the new rapidities $\{x_{n;\gamma}\}$. 
\item After calculating the expctation values of the charges ${\mathcal I}_j'$ 
 accept the new eigenstate with the Metropolis probability:
%
\begin{equation}
\label{metropolis}
\textrm{Min}\left(1,\frac{L-2M'+1}{L-2M+1}e^{-\sum_j\lambda_j({\mathcal I}'_j-
{\mathcal I}^{}_j)}\right).
\end{equation}
%
\end{enumerate}
%
In~\eqref{metropolis} the factor in front of the exponential takes into account 
that eigenstates in the same $SU(2)$ mutliplet have the same charges expectation 
value, i.e., the ${\mathcal I}_j$ are $SU(2)$ scalars. Crucially, the steps $1$ 
and $2$ are necessary in order to account for the density of states of the 
Heisenberg spin chain. The steps $1-3$ define a Markov chain, which, after a   
thermalization, generates eigenstates sampled according to~\eqref{rho-gge}. 
Notice that it is straighforward to extend the algorithm to consider fixed 
particle number $M$. More interestingly, by trivially modifying~\eqref{metropolis}
 it is possible to simulate more exotic ensembles in whichl instead of the charges 
${\mathcal I}_j$, one conseiders arbitrary functions of the rapidities. For instance, 
this would be useful in order to explore the effect of quasi-local charge on 
the $GGE$. 

We should mention that a similar method has been developed in Ref... to construct 
the the Gibbs ensemble in the Heisenberg spin chain. 

%%%%%%%%%%%%%%%%%%%%%%%%%%%%%%%%%e
\begin{figure}[t]
\includegraphics*[width=0.93\linewidth]{./draft_figs/fig3}
\caption{Finite-size scaling of the GGE averages in the Heisenberg chain: Numerical results 
 obtained from the Hilbert space Monte Carlo sampling. Here the GGE is constructed including 
 ${\mathcal I}_2, {\mathcal I}_3, {\mathcal I}_4$, with Lagrange multipliers $\lambda_2=\beta,
 \lambda_3=\lambda_4=1$. (a) $\langle {\mathcal I}_2/L\rangle$ plotted versus the chain size 
 $L$ for several values of $\beta$. The dash-dotted lines are exponential fits. (b) Same as 
 in (a) for ${\mathcal I}_4$.
}
\label{fig-2}
\end{figure}
%%%%%%%%%%%%%%%%%%%%%%%%%%%%%%%%%%

The $GGE$ expectation values $\langle{\mathcal O}\rangle$ are as the average 
of the expectation values of ${\mathcal O}$ over the eigenstates $|\{x_{n;\gamma}\}
\rangle$ generated by the Monte Carlo as 
%
\begin{equation}
\label{gge-mc}
\langle{\mathcal O}\rangle=\lim\limits_{N_\textrm{mcs}\to\infty}\frac{1}
{N_\textrm{mcs}}\sum\limits_{\{x_{n;\gamma}\}}\langle \{x_{n;\gamma}\}|{\mathcal O}
|\{x_{n;\gamma}\}\rangle,
\end{equation}
%
where $N_{\textrm{mcs}}$ is the number of Monte Carlo steps, i.e. the number of 
eigenstates in the sum. 



%#############-- GGE FOR LOCAL OBSERVABLES --########################################
\paragraph*{Local observables.---} 

The validity of the Monte Carlo method is illustrated in Fig.~\ref{fig1} considering 
the $GGE$ expectation values of the charge densities $\langle {\mathcal I}_j/L\rangle$ 
(panels (a)-(c) in the Figure) and the variance of their ensemble fluctuations 
$\sigma^2({\mathcal I}_j)\equiv \langle{\mathcal I}^2_j\rangle-\langle{\mathcal I}_j\rangle^2$
(panels (d)-(f)). Finally, panels (g)(h) plot the total magnetization $\langle S_z\rangle$ 
(i.e., the particle number) and the spin susceptibility $\chi$ (particle number fluctutations). 
Notice that $\sigma^2({\mathcal I}_2)$ is related to the specific heat, ${\mathcal I}_3\equiv
\sum_{\alpha\beta\gamma}\epsilon_{\alpha\beta\gamma}\sigma^\beta_i\sigma^\gamma_{i+1}
\sigma^\alpha_{i+2}$ is the energy current, and $\sigma^2({\mathcal I}_3)$ is related 
to the energy Drude weight. Here the data are for the truncated TGGE constructed with the first 
three charges ${\mathcal I}_2,{\mathcal I}_3,{\mathcal I}_4$. We consider several values of 
the Lagrange multipliers, namely $\lambda_3=\lambda_4=0$ (Gibbs ensemble, circles in the 
Figure), $\lambda_3=1$ and $\lambda_4=0$ (squares), and $\lambda_3=\lambda_4=1$ (rhombi). 
All our results are plotted versus the inverse temperature $\lambda_2=\beta$. The data 
are Monte Carlo results for $N_{\textrm{mcs}}=5\cdot 10^5$. In most of the cases, 
especially for small $\beta$ the Monte Carlo error bars are small than the symbols. 
As expected, the different ensemble give different expectation values, implying that 
the local observables we consider are able to distinguish different $GGE$s. Notice that 
in panel (b) $\langle {\mathcal I}_3\rangle=0$ for the Gibbs ensemble due to the parity 
invariance of ${\mathcal I}_j$ with even $j$, while in (d) $\langle S_z\rangle=0$ 
due to the $SU(2)$ symmetry of~\eqref{xxx-ham}. 
In all the panels in Fig.~\ref{fig1} the continuous lines are the analyitic results obtained 
in the thermodynamic limit by solving the $GTBA$ equations. These which fully match the Monte 
Carlo data, which signals  that the finite-size effects are negligible already for $L=16$, 
at least for the values of the $\lambda_j$ considered. 

%%%%%%%%%%%%%%%%%%%%%%%%%%%%%%%%%e
\begin{figure*}[t]
\includegraphics*[width=0.99\linewidth]{./draft_figs/fig2}
\caption{The rapidity densities $\rho_n(x)$ (for $n=1,2,3$) for the infinite temperature 
 Gibbs (panels (a)-(c)) and the GGE equilibrium states (panels (d)-(f)): Numerical  
 results for the Heisenberg spin chain obtained using the Hilbert space Monte Carlo 
 sampling. Here the GGE is constructed including only ${\mathcal I}_2$ and ${\mathcal I}_4$ 
 with fixed Lagrange multipliers $\lambda_2=0$ and $\lambda_4=1$. In all the panels the 
 data are the histograms of the $n$-strings rapidities sampled in the Monte Carlo.
 The width of the histogram bins is $\Delta x=2/L$. In each panel different histograms 
 correspond to different chain sizes $L$. All the histograms are divided by $10^3$ for 
 convenience. In (b) the arrow is to highlight the finite-size effects. In panels (a)-(c) 
 the lines are the Thermodynamic Bethe Ansatz (TBA) results. (g) Finite-temperature 
 effects: Monte Carlo data for $\rho^{\textrm{Gibbs}}_1$ for different values of the 
 inverse temperature $\beta$.
}
\label{fig1}
\end{figure*}
%%%%%%%%%%%%%%%%%%%%%%%%%%%%%%%%%%

%#############-- FINITE-SIZE CORRECTIONS --########################################

The finite-size corrections are more carefully investigated in Fig.~\ref{fig-2}.  
Fig.~\ref{fig-2} plots $\langle{\mathcal I}_2\rangle$ and $\langle {\mathcal I}_4
\rangle$ (panels (a) and (b), respectively) versus $\beta$. Here we focus on 
the $TGGE$ with $\lambda_2=\beta,\lambda_3=0$ and $\lambda_4=1$. Panel (a) 
demonstrates that finite-size effects decay exponentially with $L$ for any 
$\beta$. Clearly, corrections are larger at lower temperature, as expected.  
Moreover, they increase with the range of the operator as shown in panel (b), 
although the behavior remains exponential. 

%#############-- EXTRACTING THE ROOT DENSITIES --###################################
\paragraph*{Extracting the rapidity densities.---}

In the thermodynamic limit in each $n$-string sector the roots of~\eqref{bt-eq} become 
dense. Thus, each eigenstate is characterized by the root distribution $\pmb{\rho}\equiv
\{\rho_n\}_{n=1}^\infty$. Formally, the $\rho_n$ are defined as $\rho_n=\lim_{L\to\infty}
[L(x_{n;\gamma+1}-x_{n;\gamma})]^{-1}$. For a generic observable ${\mathcal O}$, 
the $GGE$ average becomes a functional integral as 
%
\begin{equation}
\label{th-limit}
\textrm{Tr}\big\{\exp\big({\lambda_j{\mathcal I}_j}\big){\mathcal O}\big\}
\rightarrow\int{\mathcal D}\pmb{\rho}\exp\big(S[\pmb{\rho}]+
\lambda_j{\mathcal I}_j[\pmb{\rho}]\big){\mathcal O}[\pmb{\rho}].
\end{equation}
%
Here $S[\pmb{\rho}]$ is the Yang-Yang entropy. $S[\pmb{\rho}]$ counts the number of 
eigenstates leading to the same $\pmb{\rho}$, and it is extensive. In~\eqref{th-limit} 
${\mathcal O}$ is assumed that ${\mathcal O}$ becomes a smooth function of $\pmb{\rho}$ 
in the thermodynamic limit. Since both $S$ and ${\mathcal I}_j$ are extensive, the integral 
in~\eqref{th-limit} is dominated by the saddle point $\pmb{\rho}^{sp}$, with 
$\delta(S+\lambda_j{\mathcal I}_j)/\delta\pmb{\rho}|_{\pmb{\rho}=\pmb{\rho}^{sp}}=0$.
Here $\pmb{\rho}^{sp}$ acts as a representative state for the ensemble, and it 
contains the full information about the GGE equilibrium steady state. Eq.~\eqref{gge-mc} 
suggests/implies that the representative state root densities $\rho_n^{sp}$ can be obtained 
the histograms of the roots $x_{n;\gamma}$ sampled in the Monte Carlo history, in the limit 
$L\to\infty$.  

This is supported in Fig.~\ref{fig1} considering several GGEs. Panels (a)-(c) plot the  
root densities $\rho^{sp}_n(x)$ for $n=1,2,3$ as a function of $x$ for the representative 
state (saddle point) of the infinite-temperature Gibbs ensemble. In each panel 
the different histograms correspond to different chain sizes $18\le L\le 30$. The data are 
obtained from Monte Carlo histories with $4\cdot 10^5$ Monte Carlo steps. The width of 
the histogram bins is varied with che chain size as $2/L$. In all the panels the full lines 
are the analytic results obtained from the Thermodynamic Bethe Ansatz (TBA) 
(cf.~\eqref{high-t-rho}). Remarkably, the Monte Carlo data are in good agreement with 
the TBA results. This agreement is perfect for $n=1$, whereas it become progressively 
worse upon considering larger $n>1$ (see panels (b)(c)). Clearly, the deviations from the 
TBA result vanish upon increasing the system size (see for instance the arrow in panel (b)). 
These finite-size effects are larger on the tails of the distributions. This is expected 
since large rapidities  correspond to large quasi-momenta, which are more sensitive to the 
lattice effects. Finally, finite-size effects increase with $n$, i.e., with the bound 
state sizes. The finite-temperature Gibbs ensemble is discuss in Fig.~\ref{fig1} (g), 
focusing on $\beta=1/2$ and $\beta=1$ (the different histograms in the panel). Only results 
for $\rho_1(x)$, for a chain with $L=30$ are presented. The infinite temperature histogram is 
reported for comparison. The continuous lines are now the analytic results obtained by 
solving the finite-temperature TBA equations and perfectly agree with the Monte Carlo data. 
Upon lowering the temperature the height of the peak at $x=0$ increases. This reflects 
that at $\beta=\infty$ the tail of the root distributions vanish exponentially, whereas 
for $\beta=0$ they are $\sim 1/x^4$. 

Finally, panels (d)-(f) plot $\rho_n(x)$ for the GGE ensemble. Specifically, we focus on the 
TGGE with the two charges ${\mathcal I}_2,{\mathcal I}_4$ with $\lambda_2=0$ and $\lambda_4=1$. 
In contrast with the thermal case (see (a)) $\rho_1$ exhibits a double peak structure. Similar 
to the infite-temperature Gibbs ensemble ((a)-(c) in the Figure), the data suggest that 
for $L=30$ finite-size effects are negligible, at least for $-2\le x\le 2$. 


%#########-- CONCLUSIONS --################################################
\section{Conclusions}


%########################################################################
\section{The string root densities at infinite temperature}

For infinite temperature the densities $\rho_n$ are given as 
%
\begin{equation}
\rho_n(x)=\frac{2}{\pi}\frac{1}{(n^2+x^2)(x^2+(2+n)^2)}
\end{equation}
%
Notice that 
%
\begin{equation}
\int_{-\infty}^{+\infty}\rho_n(x) dx=\frac{1}{n(n+1)(n+2)}
\end{equation}
%
Icluding the first order correction to the infinite temperature result 
one obtains  
%
\begin{multline}
\label{high-t-rho}
\rho_n(x)=\frac{2}{\pi}\frac{1}{(n^2+x^2)(x^2+(2+n)^2)}\\-
\frac{8}{\pi}\frac{n(n+2)}{(n^2+x^2)^2(x^2+(2+n)^2)^2}J\beta+{\mathcal O}
(J^2\beta^2)
\end{multline}
%


%##########################################################################
\begin{thebibliography}{99}

\bibitem{rigol-2007}
M.~Rigol, V.~Dunjko, V.~Yurovsky, and M.~Olshanii, Phys.\ Rev.\ Lett.\ 
{\bf 98}, 050405 (2007). 

\bibitem{popescu-2006}
S.~Popescu, A.~J.~Short, and A.~Winter, 
Nature\ Physics\ {\bf 2}, 754 (2006). 

\bibitem{rigol-2008}
M.~Rigol, V.~Dunjko, and M.~Olshanii, Nature {\bf 452}, 854 (2008). 

\bibitem{polkovnikov-2011}
A.~Polkovnikov, K.~Sengupta, and M.~Vengalattore, Rev.\ Mod.\ Phys.\ 
{\bf 83}, 863 (2011). 


\bibitem{eisert-2014}
J.~Eisert., M.~Friesdorf, and C.~Gogolin, arXiv:1408.5148. 


\bibitem{kollath-2007}
C.~Kollath, A.~M.~L\"auchli, and E.~Altman, Phys.\ Rev.\ Lett.\ 
{\bf 98}, 180601 (2007).

\bibitem{manmana-2007}
S.~R.~Manmana, S.~Wessel, R.~M.~Noack, and A.~Muramatsu, 
Phys.\ Rev.\ Lett.\ {\bf 98}, 210405 (2007).

\bibitem{calabrese-2007}
P.~Calabrese and J.~Cardy, J.\ Stat.\ Mech.\ P06008 (2007). 

\bibitem{cramer-2008}
M.~Cramer, C.~M.~Dawson, J.~Eisert, and T.~J.~Osborne, Phys.\ Rev.\ 
Lett.\ {\bf 100}, 030602 (2008).

\bibitem{barthel-2008}
T.~Barthel and U.~Schollw\"ock, Phys.\ Rev.\ Lett.\ {\bf 100}, 100601 
(2008). 

\bibitem{cramer-2008a}
M.~Cramer, A.~Flesch, I.~P.~McCulloch, U.~Schollw\"ock, and J.~Eisert, 
Phys.\ Rev.\ Lett.\ {\bf 101}, 063001 (2008).

\bibitem{kollar-2008}
M.~Kollar and M.~Eckstein, Phys.\ Rev.\ A {\bf 78}, 013626 (2008). 

\bibitem{iucci-2009}
A.~Iucci and M.~A.~Cazalilla, Phys.\ Rev.\ A {\bf 80}, 063619 
(2009).

\bibitem{sotiriadis-2009}
S.~Sotiriadis, P.~Calabrese, and J.~Cardy, EPL {\bf 87}, 20002 (2009). 

\bibitem{roux-2009}
G. Roux, Phys.\ Rev.\ A {\bf 79}, 021608 (2009). 

\bibitem{rigol-2009}
M.~Rigol, Phys.\ Rev.\ Lett.\ {\bf 103}, 100403 (2009).

\bibitem{rigol-2009a}
M.~Rigol,  Phys.\ Rev.\ A {\bf 80}, 053607 (2009).

\bibitem{barmettler-2009}
P.~Barmettler, M.~Punk, V.~Gritsev, E.~Demler, and E.~Altman, Phys.\ Rev.\ 
Lett.\ {\bf 102}, 130603 (2009).

\bibitem{barmettler-2010}
P.~Barmettler, M.~Punk, V.~Gritsev, E.~Demler, and E.~Altman, New\ J.\ Phys.\ 
{\bf 12}, 055017 (2010).

\bibitem{cramer-2010}
M.~Cramer and J.~Eisert, New\ J.\ Phys.\ {\bf 12}, 055020 (2010). 

\bibitem{flesch-2010}
A.~Flesch, M.~Cramer, I.~P.~McCulloch, U.~Schollw\"ock, and 
J.~Eisert, Phys.\ Rev.\ A {\bf 78}, 033608 (2008).

\bibitem{roux-2010}
G.~Roux, Phys.\ Rev.\ A {\bf 81}, 053604 (2010).

\bibitem{fioretto-2010}
D.~Fioretto and G.~Mussardo, New\ J.\ Phys.\ {\bf 12}, 
055015 (2010).

\bibitem{biroli-2010}
G.~Biroli, C.~Kollath, and A.~M.~L\"auchli, Phys.\ Rev.\ Lett.\ 
{\bf 105}, 250401 (2010). 

\bibitem{santos-2010}
L.~F.~Santos and M.~Rigol, Phys.\ Rev.\ E {\bf 82}, 031130 (2010). 

\bibitem{banuls-2011}
M.~C.~Ba\~nuls, J.~I.~Cirac, and M.~B.~Hastings, Phys.\ Rev.\ Lett.\ 
{\bf 106}, 050405 (2011). 

\bibitem{calabrese-2011}
P.~Calabrese, F.~H.~L.~Essler, and M.~Fagotti, Phys.\ Rev.\ Lett. 
{\bf 106}, 227203 (2011).

\bibitem{gogolin-2011}
C.~Gogolin, M.~P.~Mueller, and J.~Eisert, Phys.\ Rev.\ Lett.\ 
{\bf 106}, 040401 (2011).

\bibitem{rigol-2011}
M.~Rigol and M.~Fitzpatrick, Phys.\ Rev.\ A {\bf 84}, 033640 (2011).

\bibitem{caneva-2011}
T.~Caneva, E.~Canovi, D.~Rossini, G.~E.~Santoro, and A.~Silva, 
J.\ Stat.\ Mech.\ (2011) P07015. 

\bibitem{santos-2011}
L.~Santos, A.~Polkovnikov, and M.~Rigol, Phys.\ Rev.\ Lett.\ {\bf 107}, 
040601 (2011).

\bibitem{cassidy-2011}
A.~C.~Cassidy, C.~W.~Clark, and M.~Rigol, 
Phys.\ Rev.\ Lett.\ {\bf 106}, 140405 (2011). 

\bibitem{essler-2012}
F.~H.~L.~Essler, S.~Evangelisti, and M.~Fagotti, Phys.\ Rev.\ Lett.\ 
{\bf 109}, 247206 (2012). 

\bibitem{cazalilla-2012}
M.~A.~Cazalilla, A.~Iucci, and M.-C.~Chung, Phys.\ Rev.\ E {\bf 85}, 
011133 (2012). 

\bibitem{mossel-2012a}
J.~Mossel and J.-S.~Caux, New\ J.\ Phys.\ {\bf 14} 075006 (2012).

\bibitem{rigol-2012}
M.~Rigol and M.~Srednicki, Phys.\ Rev.\ Lett.\ {\bf 108}, 110601 
(2012).


\bibitem{mossel-2012}
J.~Mossel and J.-S.~Caux, J.\ Phys.\ A:\ Math.\ Theor.\ {\bf 45}, 
255001 (2012). 

\bibitem{fagotti-2013}
M.~Fagotti and F.~H.~L.~Essler, Phys.\ Rev.\ B {\bf 87}, 245107 (2013).

\bibitem{fagotti-2013a}
M.~Fagotti, Phys.\ Rev.\ B {\bf 87}, 165106 (2013). 


\bibitem{collura-2013}
M.~Collura, S.~Sotiriadis, and P.~Calabrese, Phys.\ Rev.\ Lett.\ 
{\bf 110}, 245301 (2013). 

\bibitem{caux-2013}
J.-S.~Caux and F.~H.~L.~Essler, Phys.\ Rev.\ Lett.\ {\bf 110}, 
257203 (2013). 

\bibitem{kormos-2013}
M.~Kormos, A.~Shashi, Y.-Z.~Chou, J.-S.~Caux, and A.~Imambekov, 
Phys.\ Rev.\ B {\bf 88}, 205131 (2013). 

\bibitem{bertini-2014}
B.~Bertini, D.~Schuricht, and F.~H.~L.~Essler, arXiv:1405.4813 (2014).

\bibitem{sotiriadis-2014}
S.~Sotiriadis and P.~Calabrese, J.\ Stat.\ Mech.\ (2014) P07024. 

\bibitem{essler-2014}
F.~H.~L.~Essler, S.~Kehrein, S.~R.~Manmana, and N.~J.~Robinson, Phys.\ Rev.\ 
B {\bf 89}, 165104 (2014).

\bibitem{fagotti-2014}
M.~Fagotti, M.~Collura, F.~H.~L.~Essler, and P.~Calabrese, Phys.\ Rev.\ B 
{\bf 89}, 125101 (2014).

\bibitem{fagotti-2014a}
M.~Fagotti, J.\ Stat.\ Mech.\ (2014) P03016. 

\bibitem{wouters-2014}
B.~Wouters, J.~De~Nardis, M.~Brockmann, D.~Fioretto, M.~Rigol, and 
J.-S.~Caux, Phys.\ Rev.\ Lett.\ {\bf 113}, 117202 (2014). 

\bibitem{pozsgay-2014}
B.~Pozsgay, M.~Mesty\'an, M.~A.~Werner, M.~Kormos, G.~Zar\`and, and 
G.~Tak\'acs, Phys.\ Rev.\ Lett.\ {\bf 113}, 117203 (2014).

\bibitem{greiner-2002}
M.~Greiner, O.~Mandel, T. H\"ansch, and I.~Bloch, Nature (London) 
{\bf 419}, 51 (2002). 

\bibitem{kinoshita-2006}
T.~Kinoshita, T.~Wenger, and D.~S.~Weiss, Nature (London) {\bf 440}, 
900 (2008).

\bibitem{hofferberth-2007}
S.~Hofferberth, I.~Lesanovsky, B.~Fischer, T.~Schumm, and J.~Schiedmayer, 
Nature (London) {\bf 449}, 324 (2007). 

\bibitem{bloch-2008}
I.~Bloch, J.~Dalibard, and W.~Zwerger, Rev.\ Mod.\ Phys.\ {\bf 80}, 
885 (2008).

\bibitem{trotzky-2012}
S.~Trotzky, Y.-A.~Chen, A.~Flesch, I.~P.~McCulloch, U.~Schollw\"ock, 
J.~Eisert, and I.~Bloch, Nature Phys.\ {\bf 8}, 325 (2012).

\bibitem{gring-2012}
M.~Gring, M.~Kuhnert, T.~Langen, T.~Kitagawa, B.~Rauer, M.~Schreitl, 
I.~Mazets, D.~A.~Smith, E.~Demler, and J.~Schmiedmayer, Science {\bf 337}, 
6100 (2012).

\bibitem{cheneau-2012}
M.~Cheneau, P.~Barmettler, D.~Poletti, M.~Endres, P.~Schaua, T.~Fukuhara, 
C.~Gross, I.~Bloch, C.~Kollath, and S.~Kuhr, Nature (London) {\bf 481}, 
484 (2012).

\bibitem{schneider-2012}
U.~Schneider, L.~Hackeruller, J.~P.~Ronzheimer, S.~Will, S.~Braun, T.~Best, 
I.~Bloch, E.~Demler, S.~Mandt, D.~Rasch, and A.~Rosch, Nature\ Phys.\ 
{\bf 8}, 213 (2012).

\bibitem{kunhert-2013}
M.~Kuhnert, R.~Geiger, T.~Langen, M.~Gring, B.~Rauer,
T.~Kitagawa, E.~Demler, D.~Adu Smith, and J.~Schmiedmayer, Phys.\ Rev.\ 
Lett.\ {\bf 110}, 090405 (2013).

\bibitem{langen-2013}
T.~Langen, R.~Geiger, M.~Kuhnert, B.~Rauer, and J.~Schmiedmayer, 
Nature\ Phys.\ {\bf 9}, 640 (2013).

\bibitem{meinert-2013}
F.~Meinert, M.~J.~Mark, E.~Kirilov, K.~Lauber, P.~Weinmann, 
A.~J.~Daley, and H.-C.~Nagerl, Phys.\ Rev.\ Lett.\ {\bf 111}, 
053003 (2013).

\bibitem{fukuhara-2013}
T.~Fukuhara, A.~Kantian, M.~Endres, M.~Cheneau, P.~Schaua, S.~Hild, C.~Gross, 
U.~Schollw\"ock, T.~Giamarchi, I.~Bloch, and S.~Kuhr, Nature\ Phys.\ {\bf 9}, 
235 (2013).

\bibitem{ronzheimer-2013}
J.~P.~Ronzheimer, M.~Schreiber, S.~Braun, S.~S.~Hodgman, S.~Langer, I.~P.~McCulloch, 
F. Heidrich-Meisner, I.~Bloch, and U.~Schneider, Phys.\ Rev.\ Lett.\ {\bf 110}, 
205301 (2013).

\bibitem{braun-2014}
S.~Braun, M.~Friesdorf, S.~Hodgman, M.~Schreiber, J.~Ronzheimer, A.~Riera, M.~del Rey, 
I.~Bloch, J.~Eisert, and U.~Schneider, arXiv:1403.7199.


\bibitem{deutsch-1991}
J.~M.~Deutsch, Phys.\ Rev.\ A {\bf 43}, 2046 (1991).

\bibitem{srednicki-1994}
M.~Srednicki, Phys.\ Rev.\ E {\bf 50}, 888 (1994). 

\bibitem{srednicki-1996}
M.~Srednicki, J.\ Phys.\ A {\bf 29}, L75 (1996).

\bibitem{srednicki-1999}
M.~Srednicki, J.\ Phys.\ A {\bf 32}, 1163 (1999).  

\bibitem{goldstein-2006}
S.~Goldstein, J.~L.~Lebowitz, R.~Tumulka, and N.~Zangh\'i, 
Phys.\ Rev.\ Lett.\ {\bf 96}, 050403 (2006). 

\bibitem{goldstein-2010}
S.~Goldstein, J.~L.~Lebowitz, C.~Mastrodonato, R.~Tumulka, and 
N.~Zanghi, Proc.\ R.\ Soc.\ A {\bf 466}, 3203 (2010).

\bibitem{goldstein-2010a}
S.~Goldstein, J.~L.~Lebowitz, R.~Tumulka, and N.~Zanghi, 
Eur.\ Phys.\ J.\ H {\bf 35}, 173 (2010). 

\bibitem{ikeda-2011}
T.~N.~Ikeda, Y.~Watanabe, and M.~Ueda, Phys.\ Rev.\ E {\bf 84}, 
021130 (2011). 

\bibitem{ikeda-2013} 
T.~N.~Ikeda, Y.~Watanabe, and M.~Ueda, Phys.\ Rev.\ E {\bf 87}, 
012125 (2013). 

\bibitem{steinigeweg-2013}
    R.~Steinigeweg, J.~Herbrych, and P.~Prelov\v{s}ek, Phys.\ Rev.\ E 
{\bf 87}, 012118 (2013).

\bibitem{beugeling-2013} 
W.~Beugeling, R.~Moessner, and M.~Haque, Phys.\ Rev.\ E {\bf 89}, 
042112 (2014). 

\bibitem{steinigeweg-2014}
R.~Steinigeweg, A.~Khodja, H.~Niemeyer, C.~Gogolin, and 
J.~Gemmer, Phys.\ Rev.\ Lett.\ {\bf 112}, 130403 (2014). 

\bibitem{sorg-2014}
S.~Sorg, L.~Vidmar, L.~Pollet, and F.~Heidrich-Meisner, 
arXiv:1405.5404v2. 

\bibitem{beugeling-2014} 
W.~Beugeling, R.~Moessner, and M.~Haque, arXiv:1407.2043. 

\bibitem{khemani-2014}
V.~Khemani, A.~Chandran, H.~Kim, and S.~L.~Sondhi, 
arXiv:1406.4863. 

\bibitem{kim-2014}
H.~Kim, T.~N.~Ikeda, and D.~Huse, arXiv:1408.0535. 


\bibitem{bonnes-2014}
L.~Bonnes, F.~H.~L.~Essler, and A.~M.~L\"auchli, arXiv:1404.4062 (2014).

\bibitem{caux-2011}
J.-S.~Caux and J.~Mossel, J.\ Stat.\ Mech.\ (2011) P02023. 

\bibitem{alba-2009}
V.~Alba, M.~Fagotti, and P.~Calabrese, J.\ Stat.\ Mech.\ (2009) 
P10020. 

\bibitem{kitanine-1999}
N.~Kitanine, J.~M.~Maillet, and V.~Terras, Nucl.\ Phys.\ B {\bf 554}, 
647 (1999).

\bibitem{kitanine-2000}
N.~Kitanine, J.~M.~Maillet, and V.~Terras, Nucl.\ Phys.\ B {\bf 567}, 
554 (2000).

\bibitem{amico-2008} L.~Amico, R.~Fazio, 
A.~Osterloh, and V.~Vedral, Rev.\ Mod.\ Phys.\ {\bf 80}, 517 (2008).

\bibitem{taka-book}
M.~Takahashi, {\it Thermodynamics of one-dimensional solvable models}, 
Cambridge University Press 1999. 

\bibitem{yang-1969}
C.~N.~Yang and C.~P.~Yang, J.\ Math.\ Phys. {\bf 10}, 1115 (1969).

\bibitem{takahashi-1971} 
M.~Takahashi, Prog.\ Theor.\ Phys.\ {\bf 46}, 401 (1971). 

\bibitem{grabowski-1995}
M.~P.~Grabowski and P.~Mathieu, Ann.\ Phys.\ N.Y. {\bf 243}, 
299 (1995). 

\bibitem{eisert-2009}
J. Eisert, M. Cramer, and M. B. Plenio, Rev.\ Mod.\ Phys.\ {\bf 82}, 
277 (2009). 

\bibitem{calabrese-2009}
P.~Calabrese, J.~Cardy, and B. Doyon Eds., Special issue: Entanglement 
entropy in extended systems, J.\ Phys.\ A {\bf 42}, 50 (2009).

\bibitem{cc-rev}
P.~Calabrese and J.~Cardy, J.\ Phys.\ A {\bf 42} 504005 (2009).

\bibitem{kor-book}
V.~E.~Korepin, N.~M.~Bogoliubov, and A.~G.~Izergin, \emph{Quantum 
Inverse Scattering Methods and Correlation Functions}, Cambridge 
University Press 1997. 

\bibitem{zotos-1996}
X.~Zotos and P.~Prelov\v{s}ek, Phys.\ Rev.\ B {\bf 53}, 
983 (1996).

\bibitem{castella-1996}
H.~Castella and X.~Zotos, Phys.\ Rev.\ B {\bf 54}, 4375 (1996).

\bibitem{zotos-1997}
X.~Zotos, F.~Naef, and P.~Prelov\v{s}ek, Phys.\ Rev.\ B {\bf 55}, 
11029 (1997)

\bibitem{alcaraz-2011}
F.~C.~Alcaraz, M.~I.~Berganza, and G.~Sierra, Phys.\ Rev.\ Lett.\ 
{\bf 106}, 201601 (2011).

\bibitem{pizorn-2012}
I.~Pizorn, arXiv:1202.3336. 

\bibitem{berganza-2012}
M.~I.~Berganza, F.~C.~Alcaraz, and G.~Sierra, J.\ Stat.\ Mech.\ 
(2012) P01016. 

\bibitem{wong-2013}
G.~Wong, I.~Klich, L.~A.~P.~Zayas, and D.~Vaman, 
JHEP {\bf12} (2013) 020. 

\bibitem{storms-2013}
M.~Storms, and R.~R.~P.~Singh, Phys.\ Rev.\ E {\bf 89}, 012125 
(2014). 

\bibitem{berkovits-2013}
R.~Berkovits, Phys.\ Rev.\ B {\bf 87}, 075141 (2013). 

\bibitem{essler-2013}
F.~H.~L.~Essler, A.~M.~L\"auchli, and P.~Calabrese, Phys.\ Rev.\ Lett.\ 
{\bf 110}, 115701 (2013).  

\bibitem{nozaki-2014}
M.~Nozaki, T.~Numasawa, T.~Takayanagi, Phys.\ Rev.\ Lett.\ {\bf 112}, 
111602 (2014).

\bibitem{ramirez-2014}
G.~Ramirez, J.~Rodriguez-Laguna, and G.~Sierra, arXiv:1402.5015.

\bibitem{ares-2014}
F.~Ares, J.~G.~Esteve, F.~Falceto, and E.~S\'anchez-Burillo, 
arXiv:1401.5922.

\bibitem{huang-2014}
Y.~ Huang, and J.~Moore,  arXiv:1405.1817.

\bibitem{palmai-2014}
T.~P\'almai, arXiv:1406.3182.

\bibitem{molter-2014}
J.~M\"olter, T.~Barthel,U.~Schollw\"ock, and V.~Alba, 
arXiv:1407.0066. 

\bibitem{lai-2014}
H.-H.~Lai and K.~Yang, arXiv:1409:1224


\bibitem{sato-2011}
J.~Sato, B.~Aufgebauer, H.~Boos, F.~G\"ohmann, A.~Kl\"umper, 
M.~Takahashi, and C.~Trippe, Phys.\ Rev.\ Lett.\ {\bf 106}, 257201 
(2011). 

\bibitem{fagotti-2008}
M.~Fagotti and P.~Calabrese, Phys.\ Rev.\ A {\bf 78}, 010306 (2008).

\bibitem{gurarie-2013}
V.~Gurarie, J.\ Stat.\ Mech.\ (2014) P02014. 

\bibitem{collura-2014}
M.~Collura, M.~Kormos, and P.~Calabrese, J.\ Stat.\ Mech.\ (2014) P01009. 

\bibitem{kormos-2014}
M.~Kormos, L.~Bucciantini, and P.~Calabrese, EPL {\bf 107}, 40002 (2014). 

\bibitem{caux-2005}
J.-S.~Caux and J.-M.~Maillet, Phys.\ Rev.\ Lett.\ {\bf 95}, 077201 (2005).

\bibitem{caux-2005a}
J.-S.~Caux, R.~Hagemans and J.-M.~Maillet, J.\ Stat.\ Mech.\ P09003 (2005). 

\bibitem{caux-2009}
J.-S.~Caux, J.\ Math.\ Phys.\ {\bf 50}, 095214 (2009).


\bibitem{pozsgay-2014}
B.~Pozsgay, M.~Mesty\'an, M.~A.~Werner, M.~Kormos, G.~Zar\'and, and G.~Tak\'acs,
Phys.\ Rev.\ Lett.\ {\bf 113}, 117203 (2014). 

\bibitem{wouters-2014}
B.~Wouters, M.~Brockmann, J.~De~Nardis, D.~Fioretto, M.~Rigol, and J.-S.~Caux, 
Phys.\ Rev.\ Lett.\ {\bf 113}, 117202 (2014). 

\bibitem{gu-2005}
S.-J.~Gu, N.~M.~R.~Peres, Y.-Q.~Li, Eur.\ Phys.\ J.\ B {\bf 48}, 157 (2005). 

\bibitem{ilievski-2015}
E.~Ilievski, M.~Medejak, and T.~Prosen, arXiv:1506.05049.

\bibitem{white-2004}
S.~R.~White and A.~E.~Feiguin, Phys.\ Rev.\ Lett.\ {\bf 93}, 076401 (2004).

\bibitem{daley-2004}
A.~J.~Daley, C.~Kollath, U.~Schollock, and G.~Vidal, J.\ Stat.\ Mech.\ (2004) P04005.


\end{thebibliography}

\end{document}



