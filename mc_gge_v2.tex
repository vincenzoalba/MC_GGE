\documentclass[twocolumn,superscriptaddress,prb,10pt]{revtex4-1}
%\usepackage{verbatim}
\usepackage{amsmath,amssymb}
%\usepackage{unicode-math}
\usepackage{graphicx}
\usepackage{color}
\usepackage[colorlinks,bookmarks=false,citecolor=blue,linkcolor=red,urlcolor=blue]{hyperref}
\usepackage{times}





%\usepackage[dvips]{graphics}



%%%%%%%%%%%%%   useful shortcuts %%%%%%%%%%%%%%%%%%%%%%%%%%%%%%%%%

\def \h{\hbar}   %  \h won't be used for any greek letter
\def \refe{\eqref}
\def \trm{\textrm}
\def \f{\frac}
\def \hf{\tfrac{1}{2}}    \def \HF{\dfrac{1}{2}}
\def \u{\uparrow}
\def \d{\downarrow}

\def \ord{\mathcal{O}}
\newcommand{\ra}{\rightarrow}   \newcommand{\lra}{\longrightarrow}  

\def\lba{\left(}    \def\rba{\right)}
\def\lbc{\left[}    \def\rbc{\right]}
\def\lbb{\left\{}    \def\rbb{\right\}}

\def\tr{\textrm{Tr}}
\def\refe{\eqref}

\newcommand{\bra}[1]{\langle\left.{#1}\right|}
\newcommand{\ket}[1]{\left|{#1}\right.\rangle}
\newcommand{\xpct}[1]{\langle{#1}\rangle}    % expectatn value

%\DeclareMathOperator{\tr}{tr}

%%%%%%%%%%%%%%%%%%%%%%%%%%%%%%%%%%%%%%%%%%%%%%%%%%%%%%%

\newcommand{\vp}{{\bf p}}  % usual vector quantities
\newcommand{\vq}{{\bf q}}  % double bracketing not required with \vec
\newcommand{\vk}{{\bf k}}  % but required with \bf

\renewcommand{\vr}{{\bf r}} 
\newcommand{\vx}{{\bf x}}

\newcommand{\hc}{\hat{c}}  \newcommand{\hcd}{\hat{c}^\dag} 
\newcommand{\hd}{\hat{d}}  \newcommand{\hdd}{\hat{d}^\dag} 






\begin{document}

\title{Constructing the GGE in integrable models: A Hilbert space Monte Carlo 
approach} 

\author{Vincenzo Alba}
\affiliation{International School for Advanced Studies (SISSA),
Via Bonomea 265, 34136, Trieste, Italy, 
INFN, Sezione di Trieste}

\author{Maurizio Fagotti}
\affiliation{
D\'epartement de Physique, Ecole normale superieure, CNRS, 24 rue Lhomond, 75005
Paris, France}

\date{\today}




\begin{abstract} 


\end{abstract}

% \pacs{73.43.Cd, 71.10.Pm  {\tt check!}}

\maketitle


%########################################################################
\section{Introduction}

The issue of how statistical ensembles emerge from the out-of-equilibrium 
unitary dynamics in isolated quantum many-body system is a fundamental, 
yet still challenging, problem. Much of the motivation for the renewed 
interest in this topic originated from the possibility of simulating 
efficiently the out-of-equilibrium dynamics in cold atom experiments. 
The paradigm of out-of-equilibrium experiment is the quantum quench, in 
which 



In a generic {\it global} quench in an isolated quantum many-body system let 
us consider the initial state $|\Psi_0\rangle$. In the thermodynamic limit for 
generic $|\Psi_0\rangle$ the system is expected to equilibrate. The most 
relevant question is whether expectation values of observables can be 
computed in a thermodynamic ensemble. 

Moreover, in integrable models the out-of-equilibrium dynamics is strongly  
affected by the presence of non-trivial local integral of motion. 

It is believed that the equilibrium expectation values of local operators 
at long times are described by a Generalized Gibbs Ensemble (GGE). 
More quantitatively,
%
\begin{equation}
\lim\limits_{t\to\infty}\lim\limits_{L\to\infty}\langle\Psi(t)|{\mathcal O}|
\Psi(t)\rangle=\langle{\mathcal O}\rangle_{GGE},
\end{equation}
%
with
%
\begin{equation}
\langle{\mathcal O}\rangle_{GGE}\equiv\lim\limits_{L\to\infty}\frac{\textrm{Tr}
({\mathcal O}\rho^{GGE})}{\textrm{Tr}(\rho^{GGE})}.
\end{equation}
%
with $L$ being the system size. The GGE density matrix $\rho^{GGE}$ is defined 
as 
%
\begin{equation}
\rho^{GGE}=\lim\limits_{{\mathcal N}\to\infty}
\frac{1}{Z}\exp\Big(-\sum_{j=1}^{{\mathcal N}}\lambda_j{\mathcal I}_j\Big)
\label{rho-gge}
\end{equation}
%
where $Z$ is the normalization, ${\mathcal I}_j$ the integral of motion,
 $\lambda_j$ the associated Lagrange multipliers. These are determined by 
imposing that $\langle\Psi_0|{\mathcal I}_j|\Psi_0\rangle=\langle{\mathcal O}
\rangle_{GGE}$. 


We provide a Monte Carlo method to simulate the GGE statistica ensemble. 

Remarkably, this allows to extract the rapidity densities defining the 
ensemble representative state in the thermodynamic limit. 

Finite-size effects are under control and for moderately large chain sizes 
the Monte Carlo results agree very well with the thermodynamic limit 
results, especially for small rapidities. Higher rapidities are more 
sensitive to finite size effects as they reflec short length scales. 

The method allows to address local observables for which the expression in 
terms of the rapidity is known as well. 

An alternative numerical method would be Quantum Monte Carlo. However, 
this would require the implementation of the higher conserved charges, 
other than the Hamiltonian. As the range of higher charges becomes large 
this is not easily doable in practice. 

We provide the first numerical verification of the validity of the GTBA 
equations for the Heisenberg spin chain. 

Interestingly, finite-size corrections are generically small, i.e., expontially 
decaying with the chain size. 

Our Monte Carlo approach can be trivially extended to include other conserved 
charges, both local and non-local, and arbitrary functions of the rapidities. 
While the former is related to the GGE average of the particle number, the 
latter measures its fluctuations. Notice that due to the $SU(2)$ symmetry 
of the conserved charges one has that $\langle S_z\rangle=0$ (panel (g)). 




%########################################################################
\section{The model and the method}


%########################################################################
\subsection{The Heisenberg spin chain}

The isotropic spin-$\frac{1}{2}$ Heisenberg ($XXX$) chain is defined by the 
Hamiltonian 
%
\begin{align}
\label{xxx-ham}
{\mathcal H}\equiv J\sum\limits_{i=1}^L\left[\frac{1}{2}(S_i^+S^-_{i+1} 
+S_i^{-}S_{i+1}^+)+S_i^zS_{i+1}^z\right],  
\end{align}
%
where $S^{\pm}_i\equiv (\sigma_i^x\pm i\sigma_i^y)/2$ are spin operators 
acting on the site $i$ of the chain, $S_i^z\equiv\sigma_i^z/2$, and 
$\sigma^{x,y,z}_i$ the Pauli matrices. We fix $J=1$ and use periodic 
boundary conditions identifying sites $L+1$ and $1$. Both the total spin 
$S_T^2\equiv(\sum_i \vec S_i)^2$ and the total magnetization $S_{T}^z\equiv
\sum_iS_i^z=L/2-M$, with $M$ number of down spins (particles), commute 
with~\eqref{xxx-ham}. Thus, the eigenstates of~\eqref{xxx-ham} can be 
labelled by $M$. 

Each eigenstate of~\eqref{xxx-ham} is univocally identified by a set of $M$ 
complex parameters (so-called rapidities) $\{x_\alpha\in\mathbb{C}
\}_{\alpha=1}^M$. 

In the 
thermodynamic limit $L\to\infty$ the rapidities ${x_\alpha}$ form 
``string'' patterns along the imaginary direction in the complex plane 
(string hypothesis). Specifically, the rapidities forming a string of 
length $1\le n\le M$ (so-called $n$-string) are parametrized as 
%
\begin{equation}
x_\gamma^{(n,j)}=x_\gamma^{(n)}-i(n-1-2j),\quad j=0,1,\dots,n-1,
\label{string-hyp}
\end{equation}
% 
where $x_\gamma^{(n)}\mathbb{R}$ is the real part of the string (string 
center), $j$ labels the different rapidities in the same $n$-string, and
$\gamma$ denotes strings of the same length but with different centers.

One should stress that for finite-size chains~\eqref{string-hyp} is not 
correct and deviations from the string pattern appear. However, these 
deviations typically decay exponentially with system size. 




%########################################################################
\subsection{The string hypothesis}

An important feature of the Bethe equations~\eqref{ba-eq} is that Here the scattering phases $\Theta_{m,n}$ are defined as 
%
\begin{widetext}
\begin{eqnarray}
\nonumber\Theta_{m,n}(x)\equiv\left\{\begin{array}{cc}
\vartheta\big(\frac{x}{|n-m|}\big)+\!\!\!\!\!\sum
\limits_{r=1}^{(n+m-|n-m|-1)/2}\!\!\!\!\!2\vartheta\big(\frac{x}
{|n-m|+2r}\big)+\vartheta\big(\frac{x}{n+m}\big) & \quad\mbox{if}
\quad n\ne m\\\sum\limits_{r=1}^{n-1}2\vartheta\big(\frac{x}{2r}\big)+
\vartheta\big(\frac{x}{2n}\big) & \quad\mbox{if}\quad n=m
\end{array}\right.,
\end{eqnarray}
\end{widetext}
%
where $\vartheta(x)\equiv 2\arctan(x)$. Similar to the Bethe quantum numbers 
the Bethe-Takahashi quantum numbers $I_{\gamma}^{(n)}$ identify the solutions 
of the Bethe-Takahashi equations. It can be shown that $I_\gamma^{(n)}$ are 
integers (half integers) if $L-\alpha_n$ is odd (even). Moreover, 
$I_\gamma^{(n)}$ obey the constraint 
%
\begin{equation}
\label{bt-bounds}
|I_\gamma^{(n)}|\le\frac{1}{2}(L-1-\sum\limits_{m=1}^Mt_{mn}\alpha_m),
\end{equation}
%
where $t_{mn}\equiv 2\textrm{min}(m,n)-\delta_{mn}$. 
Given a set of solutions of~\eqref{bt-eq}, and correspondingly an eigenstate 
of~\eqref{xxx-ham}, we define its ``string content'' as ${\mathcal S}\equiv
\{\mu_1,\mu_2,\dots,\mu_M\}$, with $\mu_n$ the number of $n$-strings. Clearly, 
one has that $\sum_{n=1}^M\mu_n=M$. The energy and the total momentum 
associated to a given solution of the Bethe-Takahashi equations are given as 
%
\begin{align}
& E(\{I_\gamma^{(n)}\})=-L/4+\sum\limits_{n,\gamma}\frac{2n}{
(x_\gamma^{(n)})^2+n^2}\\
& K(\{I_\gamma^{(n)}\})=\sum\limits_{n,\gamma}\frac{2\pi 
I_\gamma^{(n)}}{L}
\end{align}
%

%########################################################################
\subsection{The thermodynamic limit}

In the thermodynamic limit, i.e., for $L\to\infty$ the Bethe eigenstates are 
described by the distributions of the string centers $\rho_n$. 
These are defined such that $L\rho_n(x)$ gives the number of $n$-strings in the 
interval $[x,x+dx]$. More formally $\rho^p_n(x)$ can be defined as 
%
\begin{equation}
\label{rho-n}
\rho^p_n(x)\equiv\frac{1}{L(x_{\gamma+1}^{(n)}-x_\gamma^{(n)})}
\end{equation}
%
The Bethe-Takahashi equations~\eqref{bt-eq} are replaced by the integral 
equations 
%
\begin{equation}
\rho^p_n(x)+\rho_n^h(x)=a_n(x)+\sum\limits_{m=1}^\infty a_{nm}\*\rho^p_m(x)
\end{equation}
%

Moreover, sum over eigenstates can be replaced by 
%
\begin{equation}
\sum\limits_{\textrm{states}}\rightarrow\int{\mathcal D}
\pmb{\rho}^pe^{S_{YY}[\pmb{\rho}^p]},
\end{equation}
%
where $\pmb{\rho}^p\equiv\{\rho^p_n\}_{n=1}^\infty$, and $S_{YY}[\pmb{\rho}]$ 
is the Yang-Yang entropy
%
\begin{multline}
S_{YY}[\pmb{\rho}]\equiv N\sum\limits_{n=1}^\infty\int_{-\infty}^{+\infty}
\rho_n(x)\log\Big(1+\frac{\rho_n^h(x)}{\rho_n(x)}\Big)\\
+\rho_n^h\log\Big(1+\frac{\rho_n(x)}{\rho_n^h(x)}\Big).
\end{multline}
%
The Yang-Yang entropy counts the number of eigenstates that in the 
thermodynamic limit lead to the same $\pmb{\rho}$. 




%########################################################################
\subsection{Hilbert space structure}


For the $XXX$ chain with $L$ sites the total number of eigenstates with 
fixed number of particles $M$ is given as $C_M^L-C_{M-1}^L$, with 
$C_y^x\equiv x!/(y!(x-y)!)$ the binomial coefficient. Thus, the probability 
of an eigenstate of the $XXX$ chain with $M$ particles is given as 
$(C_M^L-C_{M-1}^L)/C_{L/2}^L$. 

For fixed $M$ the total number of eigenstates $D({\mathcal S})$ corresponding 
to a string content ${\mathcal S}\equiv\{\mu_1,\mu,\dots,\mu_M\}$ is given as 
%
\begin{equation}
\label{str-prob}
D({\mathcal S})=\prod_{i=1}^MC_{\mu_i}^{L-\sum_{j=1}^M t_{ij}\mu_j}.
\end{equation}
%

%########################################################################
\subsection{The conserved charges}

Besides the total magnetization and the momentum, the $XXX$ chain has non-trivial 
{\it local} conservation laws, due to the integrability. These extra conserved 
charges ${\mathcal I}_j$ can be obtained as 
%
\begin{equation}
\label{I-def}
\left.{\mathcal I}_{j+1}\equiv\frac{i}{(j-1)!}\frac{d^j}{dx^j}\log(\Lambda)
\right|_{x=i}.
\end{equation}
%
Here $\Lambda$ is the eigenvalue of the quantum transfer matrix $T(x)$ of the $XXX$ 
chain, with $x$ the spectral parameter. Notice that $\Lambda(x)$ depends on the 
solutions of the Bethe equations~\eqref{ba-eq}  $\{x_\alpha\}$  and it is given 
as 
%
\begin{multline}
\label{Lambda}
\Lambda\equiv\Big(\frac{x+i}{2}\Big)^L\prod\limits_\alpha\frac{x-x_\alpha-2i}{x-
x_\alpha}\\+\Big(\frac{x-i}{2}\Big)^L\prod\limits_\alpha\frac{x-x_\alpha+2i}
{x-x_\alpha}. 
\end{multline}
%
Notice that the second term in~\eqref{Lambda} does not contribute to the charges 
(cf.~\eqref{I-def}), as long as $j<L$. It is trivial to check that ${\mathcal I}_1
\equiv K$ is the total momentum, while ${\mathcal I}_2\equiv {\mathcal H}$. Moreover, 
the range of the conserved charge ${\mathcal I}_j$ is $j$, meaning that ${\mathcal I}_j$ 
acts non-trivially only on a block of $j$ adjacent spins. 


%########################################################################
\section{The Hilbert space Monte Carlo sampling}

Here we describe the Hilbert space Monte Carlo approach to simulate the GGE 
ensemble. The method allows to calculate correlation functions and to extract 
the rapidity densities $\rho_n(x)$ of the ensemble representative state. 

The basic idea of the method is to sample the Hilbert space, i.e., the eigenstates,
 of the $XXX$ chain using a Metropolis algorithm. 

Here we focus on a chain of length $L$. Given an eigenstate of the $XXX$ chain 
in the sector with $M$ particles and identified by a Bethe-Takahashi quantum number 
configuration ${\mathcal C}$, the basic Monte Carlo update generates another 
configuration ${\mathcal C}'$, and consists of five steps:
%
\begin{enumerate}
%
\item Select a particle number sector $0\le M\le L/2$ with probability 
%
\begin{equation}
P_{M'}=\frac{C^{L}_{M'}-C^L_{M'-1}}{C^{L}_{L/2}}.
\end{equation}
%
\item For fixed $M'$ select the string content ${\mathcal S}=\{\mu_1,\mu_2,\dots,
\mu_{M'}\}$ with probability
%
\begin{equation}
P_{\mathcal S}=\frac{D({\mathcal S})}{C^L_{L/2}},
\end{equation}
%
where $D({\mathcal S})$ given in~\eqref{str-prob}.
\item Generate randomly a Bethe-Takahashi quantum number configuration ${\mathcal C'}$ 
consistent with the string content ${\mathcal S}$ obtained in step $2$.
\item Solve the Bethe-Takahashi equations~\eqref{bt-eq} for using the quantum number 
configuration ${\mathcal C}'$ obtained in step $3$. Obtain the value of the conserved 
charges ${\mathcal I}'_j$ with $j=2,m$.
\item Accept the new configuration with the Metropolis probability $T({\mathcal C}
\rightarrow){\mathcal C}'$ 
%
\begin{equation}
\label{metropolis}
T=
\textrm{Min}\left(1,\frac{L-2M'+1}{L-2M+1}e^{-\sum_j\lambda_j({\mathcal I}'_j-{\mathcal 
I}^{}_j)}\right).
\end{equation}
Here $\lambda_j$ are the Lagrange multipliers, and ${\mathcal I}_j$ is the expectation 
value of the conserved charge on the eigenstate identified by the Bethe-Takahashi quantum 
numbers ${\mathcal C}$. 
\end{enumerate}
%
In~\eqref{metropolis} the factor $L-2M+1$ corresponds to the $S_z$ degeneracy in the 
sector with $M$ particles. Crucially, the Metropolis acceptance probability~\eqref{metropolis} 
relies on the fact that the conserved charges ${\mathcal I}_j$ are $SU(2)$ scalars. 

The procedure outlined above defines a Markov chain. After a thermalization time the 
configurations generated are distributed according to the $GGE$ probability~\eqref{rho-gge}. 

Notice that step $1$ and $2$ take into account the density of the states of the $XXX$ 
chain, while step $5$ generates the correct $GGE$ probability~\eqref{rho-gge}. 

%########################################################################
\section{Extracting the rapidity densities}

%%%%%%%%%%%%%%%%%%%%%%%%%%%%%%%%%e
\begin{figure*}[t]
\includegraphics*[width=0.99\linewidth]{./draft_figs/fig2}
\caption{The rapidity densities $\rho_n(x)$ (for $n=1,2,3$) for the infinite temperature 
 Gibbs (panels (a)-(c)) and the GGE equilibrium states (panels (d)-(f)): Numerical  
 results for the Heisenberg spin chain obtained using the Hilbert space Monte Carlo 
 sampling. Here the GGE is constructed including only ${\mathcal I}_2$ and ${\mathcal I}_4$ 
 with fixed Lagrange multipliers $\lambda_2=0$ and $\lambda_4=1$. In all the panels the 
 data are the histograms of the $n$-strings rapidities sampled in the Monte Carlo.
 The width of the histogram bins is $\Delta x=2/L$. In each panel different histograms 
 correspond to different chain sizes $L$. All the histograms are divided by $10^3$ for 
 convenience. In (b) the arrow is to highlight the finite-size effects. In panels (a)-(c) 
 the lines are the Thermodynamic Bethe Ansatz (TBA) results. (g) Finite-temperature 
 effects: Monte Carlo data for $\rho^{\textrm{Gibbs}}_1$ for different values of the 
 inverse temperature $\beta$.
}
\label{fig1}
\end{figure*}
%%%%%%%%%%%%%%%%%%%%%%%%%%%%%%%%%%

The rapidity densities $\rho_n^p$ of the ensemble representative state can be measured  
in Monte Carlo from the histograms of the roots generated in the Monte Carlo history.  

This is illustrated in Fig.~\ref{fig1} for several ensembles. In the Figure panels (a)-(c) 
plot the first three root densities $\rho_n(x)$ for $n=1,2,3$ as a function of $x$ for the 
representative state of the infinite-temperature Gibbs ensemble. We restrict ourselves to 
the interval $-6\le x\le 6$. In each panel the different histograms correspond to different 
chain sizes $18\le L\le 30$. The histograms are obtained from $4\cdot 10^5$ Monte Carlo 
steps. The width of the histogram bins is given as $2/L$. In all the panels the full lines 
are the analytic results obtained from the Thermodynamic Bethe Ansatz (TBA) 
(cf.~\eqref{high-t-rho}). Remarkably, although finite-size effects are present, the 
Monte Carlo data are in good agreement with the TBA results. This is especially true for 
$\rho_1$, whereas finite-size corrections become progressively worse upon considering 
larger $n>1$ (see panels (b)(c)). Clearly, these corrections are vanishing upon increasing 
the system size (see for instance the arrow in panel (b)). Moreover, the finite size effects 
become larger upon increasing the rapidity $x$. This is expected as larger rapidity 
correspond to larger quasi-momenta, which are more sensitive to the lattice effects. 
Finally, finite-size effects increase with $n$. This is somewhat expected since larger $n$ 
correspond to more extended many-spin bound states. 

%%%%%%%%%%%%%%%%%%%%%%%%%%%%%%%%%e
\begin{figure*}[t]
\includegraphics*[width=0.93\linewidth]{./draft_figs/fig1}
\caption{The Generalized Gibbs Ensemble (GGE) for the Heisenberg spin chain with 
 $L=16$ sites: numerical results obtained using the Hilbert space Monte 
 Carlo sampling. Only the first three conserved charges ${\mathcal I}_n$ ($n=1,2,3$), 
 with associated Lagrange multipliers $\lambda_n$, are included in the GGE. Here 
 ${\mathcal I}_2$ is the Hamiltonian and $\lambda_2\equiv\beta$ the inverse 
 temperature. In all the panels different symbols correspond to different  values 
 of $\lambda_3,\lambda_4$. The circles correspond to the Gibbs ensemble, i.e., 
 $\lambda_3=\lambda_4=0$. (a) The GGE average $\langle {\mathcal I}_2/L\rangle$ 
 plotted as a function of $\beta$. (b) Variance of the GGE fluctuations $\sigma^2(
 {\mathcal I}_2/L)\equiv \langle ({\mathcal I}_2/L)^2\rangle-\langle {\mathcal I}_2/L
 \rangle^2$ as a function of $\beta$. (c)(d) and (e)(f): Same as in (a)(b) for 
 ${\mathcal I}_3$ and ${\mathcal I}_4$, respectively. In all panels the dash-dotted 
 lines  are the analytical results obtained using the Generalized Thermodynamic Bethe 
 Ansatz (GTBA). (g) The GGE expectation value of the total magnetization 
 $\langle S_z\rangle$. Notice that $\langle S_z\rangle=0$ due to the $SU(2)$ 
 invariance of the conserved charges. (h) $\chi/\beta$ plotted versus $\beta$, with 
 $\chi$ being the magnetic susceptibility per site. 
}
\label{fig1}
\end{figure*}
%%%%%%%%%%%%%%%%%%%%%%%%%%%%%%%%%%


Finite-temperature ensemble can be also considered. This is illustrated in panel (g) 
discussing the Gibbs ensemble at inverse temperature $\beta=1/2$ and $\beta=1$ (different 
histograms in the panel). We focus on $\rho_1(x)$ The data for infinite temperature are 
reported for comparison. All the histograms are obtained for $L=30$. The full lines are now 
the analytic results obtained by solving the TBA equations and perfectly agree with the 
Monte Carlo data. Notice that upon lowering the temperature the height of the zero rapidity 
peak increases. This reflects that at $T=0$ the ground state of the antiferromagnetic 
Heisenberg chain are packed around $x=0$, and the contribution of nonzero rapidity is 
vanishing exponentially. 

Finally, in panels (d)(f) we present the rapidity densities $\rho_n(x)$ for the GGE ensemble. 
Specifically, here we focus on the GGE obtained including the two charges ${\mathcal I}_2$ and 
${\mathcal I}_4$. We fix the associated Lagrange multipliers to $\lambda_2=0$ and $\lambda_4=1$. 
Similar to the Gibbs ensemble, panels (a)(c) suggest that for $L=30$ the finite size effects 
are negligible, at least in the interval $-2\le x\le 2$. 



%########################################################################
\section{Local observables} 

We now turn to discuss the behavior of some {\it local} observables. This is discussed 
in Fig.~\ref{fig1}. Here we focus on GGE expectation values of local obervables. 
Here the GGE is constructed including the conserved charges ${\mathcal I}_j$ with $j=2,3,4$. 
In the Figure we consider several values of the associated Lagrange multipliers, namely 
$\lambda_3=0$ and $\lambda_4=0$ (Gibbs ensemble, circles in the Figure), $\lambda_3=1$ and 
$\lambda_4=0$ (squares in the Figure), and $\lambda_3=\lambda_4=1$ (rhombi). All the data 
are obtained using the Monte Carlo sampling approach for a chain with $L=16$ sites. The 
expectation values are the Monte Carlo averages of the local observables over the eigenstates 
sampled by the Monte Carlo. All the expectation values are plotted versus the inverse temperature $\lambda=\beta$. 
Panel (a) focuses on the GGE expectation value of energy density $\langle{\mathcal I}_2\rangle/L$. 
The variance of the associated ensemble fluctuations $\sigma^2({\mathcal I}_2/L)\equiv 
\langle ({\mathcal I}_2/L)^2\rangle-\langle{\mathcal I}_2/L\rangle^2$ are shown in 
panel (b). The latter are related to the specific heat of the chain. 

As expected, different GGEs (different symbols in the Figure) implies different behaviors of  
$\langle {\mathcal I}_2/L\rangle$ as a function of $\beta$. The continuous lines are the 
analytic results obtained by solving the GTBA equations. Remarkably, the latter fully 
match the Monte Carlo data, i.e., finite-size effects are negligible already at $L=16$. 

{\bf How do I reconcile this with Fig. 1 where finite size effects are visible even for 
$L=30$. Is that true that shorter correlations are less sensitive to higher momenta?}

This remains true if one considers the GGE expectation values of operators with larger 
supports. 

For instance, this is confirmed for the energy current ${\mathcal I}_3\equiv\sum_{\alpha\beta\gamma}
\epsilon_{\alpha\beta\gamma}\sigma^\beta_i\sigma^\gamma_{i+1}\sigma^\alpha_{i+2}$ in 
panels (c)(d). Notice that for the Gibbs ensemble one has $\langle {\mathcal I}_3\rangle=0$, 
due to the parity invariance. Similar behavior is observed for ${\mathcal I}_4$ in panels (e)(f). 

Finally, in panels (g)(h) we focus on the total magnetization $\langle S_z\rangle$ and the spin 
susceptibility $\chi$. 


The finite-size corrections for local observables are more carefully investigated 
in Fig.~\ref{fig3}, focusing on ${\mathcal I}_3$ and ${\mathcal I}_4$ (shown in 
panel (a) and (b), respectively). The Figure plots the GGE expectation values 
$\langle{\mathcal I}_2\rangle$ and $\langle {\mathcal I}_4\rangle$ versus $\beta$. 
Here we focus on the GGE with Lagrange multipliers $\lambda_2=\beta,\lambda_3=0$ 
and $\lambda_4=1$. Panel (a) demonstrates that finite-size effects decay 
exponentially with $L$ for any $\beta$. Notice that these corrections are 
larger at lower temperature, as expected since the correlation length 
increases with $\beta$. Finally, finite-size corrections increase with the 
range of the operator considered as shown in panel (b), although the 
behavior remains exponential. 






%########################################################################
\section{The string root densities at infinite temperature}



%%%%%%%%%%%%%%%%%%%%%%%%%%%%%%%%%e
\begin{figure}[t]
\includegraphics*[width=0.93\linewidth]{./draft_figs/fig3}
\caption{Finite-size scaling of the GGE averages in the Heisenberg chain: Numerical results 
 obtained from the Hilbert space Monte Carlo sampling. Here the GGE is constructed including 
 ${\mathcal I}_2, {\mathcal I}_3, {\mathcal I}_4$, with Lagrange multipliers $\lambda_2=\beta,
 \lambda_3=\lambda_4=1$. (a) $\langle {\mathcal I}_2/L\rangle$ plotted versus the chain size 
 $L$ for several values of $\beta$. The dash-dotted lines are exponential fits. (b) Same as 
 in (a) for ${\mathcal I}_4$.
}
\label{fig-3}
\end{figure}
%%%%%%%%%%%%%%%%%%%%%%%%%%%%%%%%%%

For infinite temperature the densities $\rho_n$ are given as 
%
\begin{equation}
\rho_n(x)=\frac{2}{\pi}\frac{1}{(n^2+x^2)(x^2+(2+n)^2)}
\end{equation}
%
Notice that 
%
\begin{equation}
\int_{-\infty}^{+\infty}\rho_n(x) dx=\frac{1}{n(n+1)(n+2)}
\end{equation}
%
Icluding the first order correction to the infinite temperature result 
one obtains  
%
\begin{multline}
\label{high-t-rho}
\rho_n(x)=\frac{2}{\pi}\frac{1}{(n^2+x^2)(x^2+(2+n)^2)}\\-
\frac{8}{\pi}\frac{n(n+2)}{(n^2+x^2)^2(x^2+(2+n)^2)^2}J\beta+{\mathcal O}
(J^2\beta^2)
\end{multline}
%


%##########################################################################
\begin{thebibliography}{99}

\bibitem{rigol-2007}
M.~Rigol, V.~Dunjko, V.~Yurovsky, and M.~Olshanii, Phys.\ Rev.\ Lett.\ 
{\bf 98}, 050405 (2007). 

\bibitem{popescu-2006}
S.~Popescu, A.~J.~Short, and A.~Winter, 
Nature\ Physics\ {\bf 2}, 754 (2006). 

\bibitem{rigol-2008}
M.~Rigol, V.~Dunjko, and M.~Olshanii, Nature {\bf 452}, 854 (2008). 

\bibitem{polkovnikov-2011}
A.~Polkovnikov, K.~Sengupta, and M.~Vengalattore, Rev.\ Mod.\ Phys.\ 
{\bf 83}, 863 (2011). 


\bibitem{eisert-2014}
J.~Eisert., M.~Friesdorf, and C.~Gogolin, arXiv:1408.5148. 


\bibitem{kollath-2007}
C.~Kollath, A.~M.~L\"auchli, and E.~Altman, Phys.\ Rev.\ Lett.\ 
{\bf 98}, 180601 (2007).

\bibitem{manmana-2007}
S.~R.~Manmana, S.~Wessel, R.~M.~Noack, and A.~Muramatsu, 
Phys.\ Rev.\ Lett.\ {\bf 98}, 210405 (2007).

\bibitem{calabrese-2007}
P.~Calabrese and J.~Cardy, J.\ Stat.\ Mech.\ P06008 (2007). 

\bibitem{cramer-2008}
M.~Cramer, C.~M.~Dawson, J.~Eisert, and T.~J.~Osborne, Phys.\ Rev.\ 
Lett.\ {\bf 100}, 030602 (2008).

\bibitem{barthel-2008}
T.~Barthel and U.~Schollw\"ock, Phys.\ Rev.\ Lett.\ {\bf 100}, 100601 
(2008). 

\bibitem{cramer-2008a}
M.~Cramer, A.~Flesch, I.~P.~McCulloch, U.~Schollw\"ock, and J.~Eisert, 
Phys.\ Rev.\ Lett.\ {\bf 101}, 063001 (2008).

\bibitem{kollar-2008}
M.~Kollar and M.~Eckstein, Phys.\ Rev.\ A {\bf 78}, 013626 (2008). 

\bibitem{iucci-2009}
A.~Iucci and M.~A.~Cazalilla, Phys.\ Rev.\ A {\bf 80}, 063619 
(2009).

\bibitem{sotiriadis-2009}
S.~Sotiriadis, P.~Calabrese, and J.~Cardy, EPL {\bf 87}, 20002 (2009). 

\bibitem{roux-2009}
G. Roux, Phys.\ Rev.\ A {\bf 79}, 021608 (2009). 

\bibitem{rigol-2009}
M.~Rigol, Phys.\ Rev.\ Lett.\ {\bf 103}, 100403 (2009).

\bibitem{rigol-2009a}
M.~Rigol,  Phys.\ Rev.\ A {\bf 80}, 053607 (2009).

\bibitem{barmettler-2009}
P.~Barmettler, M.~Punk, V.~Gritsev, E.~Demler, and E.~Altman, Phys.\ Rev.\ 
Lett.\ {\bf 102}, 130603 (2009).

\bibitem{barmettler-2010}
P.~Barmettler, M.~Punk, V.~Gritsev, E.~Demler, and E.~Altman, New\ J.\ Phys.\ 
{\bf 12}, 055017 (2010).

\bibitem{cramer-2010}
M.~Cramer and J.~Eisert, New\ J.\ Phys.\ {\bf 12}, 055020 (2010). 

\bibitem{flesch-2010}
A.~Flesch, M.~Cramer, I.~P.~McCulloch, U.~Schollw\"ock, and 
J.~Eisert, Phys.\ Rev.\ A {\bf 78}, 033608 (2008).

\bibitem{roux-2010}
G.~Roux, Phys.\ Rev.\ A {\bf 81}, 053604 (2010).

\bibitem{fioretto-2010}
D.~Fioretto and G.~Mussardo, New\ J.\ Phys.\ {\bf 12}, 
055015 (2010).

\bibitem{biroli-2010}
G.~Biroli, C.~Kollath, and A.~M.~L\"auchli, Phys.\ Rev.\ Lett.\ 
{\bf 105}, 250401 (2010). 

\bibitem{santos-2010}
L.~F.~Santos and M.~Rigol, Phys.\ Rev.\ E {\bf 82}, 031130 (2010). 

\bibitem{banuls-2011}
M.~C.~Ba\~nuls, J.~I.~Cirac, and M.~B.~Hastings, Phys.\ Rev.\ Lett.\ 
{\bf 106}, 050405 (2011). 

\bibitem{calabrese-2011}
P.~Calabrese, F.~H.~L.~Essler, and M.~Fagotti, Phys.\ Rev.\ Lett. 
{\bf 106}, 227203 (2011).

\bibitem{gogolin-2011}
C.~Gogolin, M.~P.~Mueller, and J.~Eisert, Phys.\ Rev.\ Lett.\ 
{\bf 106}, 040401 (2011).

\bibitem{rigol-2011}
M.~Rigol and M.~Fitzpatrick, Phys.\ Rev.\ A {\bf 84}, 033640 (2011).

\bibitem{caneva-2011}
T.~Caneva, E.~Canovi, D.~Rossini, G.~E.~Santoro, and A.~Silva, 
J.\ Stat.\ Mech.\ (2011) P07015. 

\bibitem{santos-2011}
L.~Santos, A.~Polkovnikov, and M.~Rigol, Phys.\ Rev.\ Lett.\ {\bf 107}, 
040601 (2011).

\bibitem{cassidy-2011}
A.~C.~Cassidy, C.~W.~Clark, and M.~Rigol, 
Phys.\ Rev.\ Lett.\ {\bf 106}, 140405 (2011). 

\bibitem{essler-2012}
F.~H.~L.~Essler, S.~Evangelisti, and M.~Fagotti, Phys.\ Rev.\ Lett.\ 
{\bf 109}, 247206 (2012). 

\bibitem{cazalilla-2012}
M.~A.~Cazalilla, A.~Iucci, and M.-C.~Chung, Phys.\ Rev.\ E {\bf 85}, 
011133 (2012). 

\bibitem{mossel-2012a}
J.~Mossel and J.-S.~Caux, New\ J.\ Phys.\ {\bf 14} 075006 (2012).

\bibitem{rigol-2012}
M.~Rigol and M.~Srednicki, Phys.\ Rev.\ Lett.\ {\bf 108}, 110601 
(2012).


\bibitem{mossel-2012}
J.~Mossel and J.-S.~Caux, J.\ Phys.\ A:\ Math.\ Theor.\ {\bf 45}, 
255001 (2012). 

\bibitem{fagotti-2013}
M.~Fagotti and F.~H.~L.~Essler, Phys.\ Rev.\ B {\bf 87}, 245107 (2013).

\bibitem{fagotti-2013a}
M.~Fagotti, Phys.\ Rev.\ B {\bf 87}, 165106 (2013). 


\bibitem{collura-2013}
M.~Collura, S.~Sotiriadis, and P.~Calabrese, Phys.\ Rev.\ Lett.\ 
{\bf 110}, 245301 (2013). 

\bibitem{caux-2013}
J.-S.~Caux and F.~H.~L.~Essler, Phys.\ Rev.\ Lett.\ {\bf 110}, 
257203 (2013). 

\bibitem{kormos-2013}
M.~Kormos, A.~Shashi, Y.-Z.~Chou, J.-S.~Caux, and A.~Imambekov, 
Phys.\ Rev.\ B {\bf 88}, 205131 (2013). 

\bibitem{bertini-2014}
B.~Bertini, D.~Schuricht, and F.~H.~L.~Essler, arXiv:1405.4813 (2014).

\bibitem{sotiriadis-2014}
S.~Sotiriadis and P.~Calabrese, J.\ Stat.\ Mech.\ (2014) P07024. 

\bibitem{essler-2014}
F.~H.~L.~Essler, S.~Kehrein, S.~R.~Manmana, and N.~J.~Robinson, Phys.\ Rev.\ 
B {\bf 89}, 165104 (2014).

\bibitem{fagotti-2014}
M.~Fagotti, M.~Collura, F.~H.~L.~Essler, and P.~Calabrese, Phys.\ Rev.\ B 
{\bf 89}, 125101 (2014).

\bibitem{fagotti-2014a}
M.~Fagotti, J.\ Stat.\ Mech.\ (2014) P03016. 

\bibitem{wouters-2014}
B.~Wouters, J.~De~Nardis, M.~Brockmann, D.~Fioretto, M.~Rigol, and 
J.-S.~Caux, Phys.\ Rev.\ Lett.\ {\bf 113}, 117202 (2014). 

\bibitem{pozsgay-2014}
B.~Pozsgay, M.~Mesty\'an, M.~A.~Werner, M.~Kormos, G.~Zar\`and, and 
G.~Tak\'acs, Phys.\ Rev.\ Lett.\ {\bf 113}, 117203 (2014).

\bibitem{greiner-2002}
M.~Greiner, O.~Mandel, T. H\"ansch, and I.~Bloch, Nature (London) 
{\bf 419}, 51 (2002). 

\bibitem{kinoshita-2006}
T.~Kinoshita, T.~Wenger, and D.~S.~Weiss, Nature (London) {\bf 440}, 
900 (2008).

\bibitem{hofferberth-2007}
S.~Hofferberth, I.~Lesanovsky, B.~Fischer, T.~Schumm, and J.~Schiedmayer, 
Nature (London) {\bf 449}, 324 (2007). 

\bibitem{bloch-2008}
I.~Bloch, J.~Dalibard, and W.~Zwerger, Rev.\ Mod.\ Phys.\ {\bf 80}, 
885 (2008).

\bibitem{trotzky-2012}
S.~Trotzky, Y.-A.~Chen, A.~Flesch, I.~P.~McCulloch, U.~Schollw\"ock, 
J.~Eisert, and I.~Bloch, Nature Phys.\ {\bf 8}, 325 (2012).

\bibitem{gring-2012}
M.~Gring, M.~Kuhnert, T.~Langen, T.~Kitagawa, B.~Rauer, M.~Schreitl, 
I.~Mazets, D.~A.~Smith, E.~Demler, and J.~Schmiedmayer, Science {\bf 337}, 
6100 (2012).

\bibitem{cheneau-2012}
M.~Cheneau, P.~Barmettler, D.~Poletti, M.~Endres, P.~Schaua, T.~Fukuhara, 
C.~Gross, I.~Bloch, C.~Kollath, and S.~Kuhr, Nature (London) {\bf 481}, 
484 (2012).

\bibitem{schneider-2012}
U.~Schneider, L.~Hackeruller, J.~P.~Ronzheimer, S.~Will, S.~Braun, T.~Best, 
I.~Bloch, E.~Demler, S.~Mandt, D.~Rasch, and A.~Rosch, Nature\ Phys.\ 
{\bf 8}, 213 (2012).

\bibitem{kunhert-2013}
M.~Kuhnert, R.~Geiger, T.~Langen, M.~Gring, B.~Rauer,
T.~Kitagawa, E.~Demler, D.~Adu Smith, and J.~Schmiedmayer, Phys.\ Rev.\ 
Lett.\ {\bf 110}, 090405 (2013).

\bibitem{langen-2013}
T.~Langen, R.~Geiger, M.~Kuhnert, B.~Rauer, and J.~Schmiedmayer, 
Nature\ Phys.\ {\bf 9}, 640 (2013).

\bibitem{meinert-2013}
F.~Meinert, M.~J.~Mark, E.~Kirilov, K.~Lauber, P.~Weinmann, 
A.~J.~Daley, and H.-C.~Nagerl, Phys.\ Rev.\ Lett.\ {\bf 111}, 
053003 (2013).

\bibitem{fukuhara-2013}
T.~Fukuhara, A.~Kantian, M.~Endres, M.~Cheneau, P.~Schaua, S.~Hild, C.~Gross, 
U.~Schollw\"ock, T.~Giamarchi, I.~Bloch, and S.~Kuhr, Nature\ Phys.\ {\bf 9}, 
235 (2013).

\bibitem{ronzheimer-2013}
J.~P.~Ronzheimer, M.~Schreiber, S.~Braun, S.~S.~Hodgman, S.~Langer, I.~P.~McCulloch, 
F. Heidrich-Meisner, I.~Bloch, and U.~Schneider, Phys.\ Rev.\ Lett.\ {\bf 110}, 
205301 (2013).

\bibitem{braun-2014}
S.~Braun, M.~Friesdorf, S.~Hodgman, M.~Schreiber, J.~Ronzheimer, A.~Riera, M.~del Rey, 
I.~Bloch, J.~Eisert, and U.~Schneider, arXiv:1403.7199.


\bibitem{deutsch-1991}
J.~M.~Deutsch, Phys.\ Rev.\ A {\bf 43}, 2046 (1991).

\bibitem{srednicki-1994}
M.~Srednicki, Phys.\ Rev.\ E {\bf 50}, 888 (1994). 

\bibitem{srednicki-1996}
M.~Srednicki, J.\ Phys.\ A {\bf 29}, L75 (1996).

\bibitem{srednicki-1999}
M.~Srednicki, J.\ Phys.\ A {\bf 32}, 1163 (1999).  

\bibitem{goldstein-2006}
S.~Goldstein, J.~L.~Lebowitz, R.~Tumulka, and N.~Zangh\'i, 
Phys.\ Rev.\ Lett.\ {\bf 96}, 050403 (2006). 

\bibitem{goldstein-2010}
S.~Goldstein, J.~L.~Lebowitz, C.~Mastrodonato, R.~Tumulka, and 
N.~Zanghi, Proc.\ R.\ Soc.\ A {\bf 466}, 3203 (2010).

\bibitem{goldstein-2010a}
S.~Goldstein, J.~L.~Lebowitz, R.~Tumulka, and N.~Zanghi, 
Eur.\ Phys.\ J.\ H {\bf 35}, 173 (2010). 

\bibitem{ikeda-2011}
T.~N.~Ikeda, Y.~Watanabe, and M.~Ueda, Phys.\ Rev.\ E {\bf 84}, 
021130 (2011). 

\bibitem{ikeda-2013} 
T.~N.~Ikeda, Y.~Watanabe, and M.~Ueda, Phys.\ Rev.\ E {\bf 87}, 
012125 (2013). 

\bibitem{steinigeweg-2013}
    R.~Steinigeweg, J.~Herbrych, and P.~Prelov\v{s}ek, Phys.\ Rev.\ E 
{\bf 87}, 012118 (2013).

\bibitem{beugeling-2013} 
W.~Beugeling, R.~Moessner, and M.~Haque, Phys.\ Rev.\ E {\bf 89}, 
042112 (2014). 

\bibitem{steinigeweg-2014}
R.~Steinigeweg, A.~Khodja, H.~Niemeyer, C.~Gogolin, and 
J.~Gemmer, Phys.\ Rev.\ Lett.\ {\bf 112}, 130403 (2014). 

\bibitem{sorg-2014}
S.~Sorg, L.~Vidmar, L.~Pollet, and F.~Heidrich-Meisner, 
arXiv:1405.5404v2. 

\bibitem{beugeling-2014} 
W.~Beugeling, R.~Moessner, and M.~Haque, arXiv:1407.2043. 

\bibitem{khemani-2014}
V.~Khemani, A.~Chandran, H.~Kim, and S.~L.~Sondhi, 
arXiv:1406.4863. 

\bibitem{kim-2014}
H.~Kim, T.~N.~Ikeda, and D.~Huse, arXiv:1408.0535. 


\bibitem{bonnes-2014}
L.~Bonnes, F.~H.~L.~Essler, and A.~M.~L\"auchli, arXiv:1404.4062 (2014).

\bibitem{caux-2011}
J.-S.~Caux and J.~Mossel, J.\ Stat.\ Mech.\ (2011) P02023. 

\bibitem{alba-2009}
V.~Alba, M.~Fagotti, and P.~Calabrese, J.\ Stat.\ Mech.\ (2009) 
P10020. 

\bibitem{kitanine-1999}
N.~Kitanine, J.~M.~Maillet, and V.~Terras, Nucl.\ Phys.\ B {\bf 554}, 
647 (1999).

\bibitem{kitanine-2000}
N.~Kitanine, J.~M.~Maillet, and V.~Terras, Nucl.\ Phys.\ B {\bf 567}, 
554 (2000).

\bibitem{amico-2008} L.~Amico, R.~Fazio, 
A.~Osterloh, and V.~Vedral, Rev.\ Mod.\ Phys.\ {\bf 80}, 517 (2008).

\bibitem{taka-book}
M.~Takahashi, {\it Thermodynamics of one-dimensional solvable models}, 
Cambridge University Press 1999. 

\bibitem{yang-1969}
C.~N.~Yang and C.~P.~Yang, J.\ Math.\ Phys. {\bf 10}, 1115 (1969).

\bibitem{takahashi-1971} 
M.~Takahashi, Prog.\ Theor.\ Phys.\ {\bf 46}, 401 (1971). 

\bibitem{grabowski-1995}
M.~P.~Grabowski and P.~Mathieu, Ann.\ Phys.\ N.Y. {\bf 243}, 
299 (1995). 

\bibitem{eisert-2009}
J. Eisert, M. Cramer, and M. B. Plenio, Rev.\ Mod.\ Phys.\ {\bf 82}, 
277 (2009). 

\bibitem{calabrese-2009}
P.~Calabrese, J.~Cardy, and B. Doyon Eds., Special issue: Entanglement 
entropy in extended systems, J.\ Phys.\ A {\bf 42}, 50 (2009).

\bibitem{cc-rev}
P.~Calabrese and J.~Cardy, J.\ Phys.\ A {\bf 42} 504005 (2009).

\bibitem{kor-book}
V.~E.~Korepin, N.~M.~Bogoliubov, and A.~G.~Izergin, \emph{Quantum 
Inverse Scattering Methods and Correlation Functions}, Cambridge 
University Press 1997. 

\bibitem{zotos-1996}
X.~Zotos and P.~Prelov\v{s}ek, Phys.\ Rev.\ B {\bf 53}, 
983 (1996).

\bibitem{castella-1996}
H.~Castella and X.~Zotos, Phys.\ Rev.\ B {\bf 54}, 4375 (1996).

\bibitem{zotos-1997}
X.~Zotos, F.~Naef, and P.~Prelov\v{s}ek, Phys.\ Rev.\ B {\bf 55}, 
11029 (1997)

\bibitem{alcaraz-2011}
F.~C.~Alcaraz, M.~I.~Berganza, and G.~Sierra, Phys.\ Rev.\ Lett.\ 
{\bf 106}, 201601 (2011).

\bibitem{pizorn-2012}
I.~Pizorn, arXiv:1202.3336. 

\bibitem{berganza-2012}
M.~I.~Berganza, F.~C.~Alcaraz, and G.~Sierra, J.\ Stat.\ Mech.\ 
(2012) P01016. 

\bibitem{wong-2013}
G.~Wong, I.~Klich, L.~A.~P.~Zayas, and D.~Vaman, 
JHEP {\bf12} (2013) 020. 

\bibitem{storms-2013}
M.~Storms, and R.~R.~P.~Singh, Phys.\ Rev.\ E {\bf 89}, 012125 
(2014). 

\bibitem{berkovits-2013}
R.~Berkovits, Phys.\ Rev.\ B {\bf 87}, 075141 (2013). 

\bibitem{essler-2013}
F.~H.~L.~Essler, A.~M.~L\"auchli, and P.~Calabrese, Phys.\ Rev.\ Lett.\ 
{\bf 110}, 115701 (2013).  

\bibitem{nozaki-2014}
M.~Nozaki, T.~Numasawa, T.~Takayanagi, Phys.\ Rev.\ Lett.\ {\bf 112}, 
111602 (2014).

\bibitem{ramirez-2014}
G.~Ramirez, J.~Rodriguez-Laguna, and G.~Sierra, arXiv:1402.5015.

\bibitem{ares-2014}
F.~Ares, J.~G.~Esteve, F.~Falceto, and E.~S\'anchez-Burillo, 
arXiv:1401.5922.

\bibitem{huang-2014}
Y.~ Huang, and J.~Moore,  arXiv:1405.1817.

\bibitem{palmai-2014}
T.~P\'almai, arXiv:1406.3182.

\bibitem{molter-2014}
J.~M\"olter, T.~Barthel,U.~Schollw\"ock, and V.~Alba, 
arXiv:1407.0066. 

\bibitem{lai-2014}
H.-H.~Lai and K.~Yang, arXiv:1409:1224


\bibitem{sato-2011}
J.~Sato, B.~Aufgebauer, H.~Boos, F.~G\"ohmann, A.~Kl\"umper, 
M.~Takahashi, and C.~Trippe, Phys.\ Rev.\ Lett.\ {\bf 106}, 257201 
(2011). 

\bibitem{fagotti-2008}
M.~Fagotti and P.~Calabrese, Phys.\ Rev.\ A {\bf 78}, 010306 (2008).

\bibitem{gurarie-2013}
V.~Gurarie, J.\ Stat.\ Mech.\ (2014) P02014. 

\bibitem{collura-2014}
M.~Collura, M.~Kormos, and P.~Calabrese, J.\ Stat.\ Mech.\ (2014) P01009. 

\bibitem{kormos-2014}
M.~Kormos, L.~Bucciantini, and P.~Calabrese, EPL {\bf 107}, 40002 (2014). 

\bibitem{caux-2005}
J.-S.~Caux and J.-M.~Maillet, Phys.\ Rev.\ Lett.\ {\bf 95}, 077201 (2005).

\bibitem{caux-2005a}
J.-S.~Caux, R.~Hagemans and J.-M.~Maillet, J.\ Stat.\ Mech.\ P09003 (2005). 

\bibitem{caux-2009}
J.-S.~Caux, J.\ Math.\ Phys.\ {\bf 50}, 095214 (2009).


\end{thebibliography}

\end{document}



