\documentclass[twocolumn,superscriptaddress,prb,10pt]{revtex4-1}
%\usepackage{verbatim}
\usepackage{amsmath,amssymb}
\usepackage{graphicx}
\usepackage{color}
\usepackage[colorlinks,bookmarks=false,citecolor=blue,linkcolor=red,urlcolor=blue]{hyperref}
\usepackage{times}





%\usepackage[dvips]{graphics}



%%%%%%%%%%%%%   useful shortcuts %%%%%%%%%%%%%%%%%%%%%%%%%%%%%%%%%

\def \h{\hbar}   %  \h won't be used for any greek letter
\def \refe{\eqref}
\def \trm{\textrm}
\def \f{\frac}
\def \hf{\tfrac{1}{2}}    \def \HF{\dfrac{1}{2}}
\def \u{\uparrow}
\def \d{\downarrow}

\def \ord{\mathcal{O}}
\newcommand{\ra}{\rightarrow}   \newcommand{\lra}{\longrightarrow}  

\def\lba{\left(}    \def\rba{\right)}
\def\lbc{\left[}    \def\rbc{\right]}
\def\lbb{\left\{}    \def\rbb{\right\}}

\def\tr{\textrm{Tr}}
\def\refe{\eqref}

\newcommand{\bra}[1]{\langle\left.{#1}\right|}
\newcommand{\ket}[1]{\left|{#1}\right.\rangle}
\newcommand{\xpct}[1]{\langle{#1}\rangle}    % expectatn value

%\DeclareMathOperator{\tr}{tr}

%%%%%%%%%%%%%%%%%%%%%%%%%%%%%%%%%%%%%%%%%%%%%%%%%%%%%%%

\newcommand{\vp}{{\bf p}}  % usual vector quantities
\newcommand{\vq}{{\bf q}}  % double bracketing not required with \vec
\newcommand{\vk}{{\bf k}}  % but required with \bf

\renewcommand{\vr}{{\bf r}} 
\newcommand{\vx}{{\bf x}}

\newcommand{\hc}{\hat{c}}  \newcommand{\hcd}{\hat{c}^\dag} 
\newcommand{\hd}{\hat{d}}  \newcommand{\hdd}{\hat{d}^\dag} 






\begin{document}

\title{Constructing the GGE in integrable models: A Hilbert space Monte Carlo 
approach} 

\author{Vincenzo Alba}
\affiliation{International School for Advanced Studies (SISSA),
Via Bonomea 265, 34136, Trieste, Italy, 
INFN, Sezione di Trieste}

\author{Maurizio Fagotti}
\affiliation{
D\'epartement de Physique, Ecole normale superieure, CNRS, 24 rue Lhomond, 75005
Paris, France}

\date{\today}




\begin{abstract} 


\end{abstract}

% \pacs{73.43.Cd, 71.10.Pm  {\tt check!}}

\maketitle


%########################################################################
\section{Introduction}


%
\begin{equation}
\rho^{GGE}=\frac{1}{Z}\exp\Big(-\sum_j\lambda_j{\mathcal I}_j\Big)
\label{gge}
\end{equation}
%

%########################################################################
\section{The model and the method}

\subsection{The Heisenberg spin chain}

The isotropic spin-$\frac{1}{2}$ Heisenberg ($XXX$) chain is defined by the 
Hamiltonian 
%
\begin{align}
\label{xxx_ham}
{\mathcal H}\equiv J\sum\limits_{i=1}^L\left[\frac{1}{2}(S_i^+S^-_{i+1} 
+S_i^{-}S_{i+1}^+)+S_i^zS_{i+1}^z\right],  
\end{align}
%
where $S^{\pm}_i\equiv (\sigma_i^x\pm i\sigma_i^y)/2$ are spin operators 
acting on the site $i$ of the chain, $S_i^z\equiv\sigma_i^z/2$, and 
$\sigma^{x,y,z}_i$ the Pauli matrices. We fix $J=1$ and use periodic 
boundary conditions identifying sites $L+1$ and $1$. 
Both the total spin $S_T^2\equiv(\sum_i \vec S_i)^2$ and the total 
magnetization $S_{T}^z\equiv\sum_iS_i^z=L/2-M$, with $M$ being the 
number of down spins, commute with~\eqref{xxx_ham}. Thus, the 
eigenstates of~\eqref{xxx_ham} can be labelled by $M$. Following the 
Bethe ansatz literature here we refer to the down spins as particles. 

Each eigenstate of~\eqref{xxx_ham} is univocally identified by a set of $M$ 
complex parameters (the so-called rapidities) $\{x_\alpha\}$ with 
$\alpha=1,\dots,M$. These are solutions of a set of non linear algebraic 
equations, the Bethe equations
%
\begin{equation}
\arctan(x_\alpha)=\frac{\pi}{L}J_\alpha+
\frac{1}{L}\sum\limits_{\beta\ne \alpha}\arctan\Big(\frac{
x_\alpha-x_\beta}{2}\Big).   
\label{ba_eq}
\end{equation}
%
Here $J_\alpha\in\frac{1}{2}\mathbb{Z}$ are the Bethe quantum numbers. 
Any choice of $-L/2\le J_\alpha\le L/2$ identifies a solution of the 
Bethe equations and an eigenstate of the XXX chain. 

Besides the the total magnetization and the momentum, the $XXX$ chain 
has non-trivial conservation laws, due to the integrability. These extra 
conserved quantities ${\mathcal I}_j$ are obtained as 
%
\begin{equation}
\left.{\mathcal I}_j\equiv\frac{i}{(j-1)!}\frac{d^j}{dx^j}\log(\Lambda)
\right|_{x=i}.
\end{equation}
%
Here $\Lambda$ in the Algebraic Bethe Ansatz approach is the eigenvalue 
of the transfer matrix of the $XXX$ chain. This depends on a parameter 
$x$ and on the set of rapidities $\{x_\alpha\}$  and it 
is given as 
%
\begin{multline}
\Lambda(x,\{x_\alpha\})\equiv\Big(\frac{x+i}{2}\Big)^L
\prod\limits_\alpha\frac{x-x_\alpha-2i}{x-x_\alpha}\\
+\Big(\frac{x-i}{2}\Big)^L
\prod\limits_\alpha\frac{x-x_\alpha+2i}{x-x_\alpha}. 
\end{multline}
%
An important feature of the XXX chain is that in the thermodynamic limit 
$L\to\infty$, the solutions of~\eqref{ba_eq} form ``string'' pattern in 
the complex plane (string hypothesis). Specifically, the rapidities 
forming a string of length of length $1\le n\le M$ ($n$-string) are 
parametrized as 
%
\begin{equation}
x_\gamma^{(n,j)}=x_\gamma^{(n)}-i(n-1-2j),\quad j=0,1,\dots,n-1.
\label{string_hyp}
\end{equation}
% 
Here $x_\gamma^{(n)}\mathbb{R}$ is the real part of the string 
(string center), and $\gamma$ labels strings of the same length but with 
different centers.

Substituting the string hypothesis in~\eqref{ba_eq} one obtains a set of 
the discrete Bethe-Takahashi equations for the string centers as 
%
\begin{equation}
2L\arctan(x_\gamma^{(n)}/n)=2\pi I_\gamma^{(n)}+\sum\limits_{(m,\beta)
\ne(n,\gamma)}\Theta_{m,n}(x_\gamma^{(n)}-x_\beta^{(m)}).
\label{bt-eq}
\end{equation}
%
Here the scattering phases $\Theta_{m,n}$ are defined as 
%
\begin{widetext}
\begin{eqnarray}
\nonumber\Theta_{m,n}(x)\equiv\left\{\begin{array}{cc}
\vartheta\big(\frac{x}{|n-m|}\big)+\!\!\!\!\!\sum
\limits_{r=1}^{(n+m-|n-m|-1)/2}\!\!\!\!\!2\vartheta\big(\frac{x}
{|n-m|+2r}\big)+\vartheta\big(\frac{x}{n+m}\big) & \quad\mbox{if}
\quad n\ne m\\\sum\limits_{r=1}^{n-1}2\vartheta\big(\frac{x}{2r}\big)+
\vartheta\big(\frac{x}{2n}\big) & \quad\mbox{if}\quad n=m
\end{array}\right.
\end{eqnarray}
\end{widetext}
%
Here $\vartheta(x)\equiv 2\arctan(x)$. Similar to the Bethe quantum numbers 
the Bethe-Takahashi quantum numbers $I_{\gamma}^{(n)}$ identify the solutions 
of the Bethe-Takahashi equations. We denote as $\alpha_n$ the number of 
strings of length $n$ in the rapidities identifying an eigenstate. 

Clearly, for a generic eigenstate one has that $\sum_{n=1}^M\alpha_n=M$. 

It can be shown that $I_\gamma^{(n)}$ are integers (half integers) if 
$L-\alpha_n$ is odd (even). 

The energy and the total momentum associated to a given solution of the 
Bethe-Takahashi equations are given as 
%
\begin{align}
& E(\{x_\gamma^{(n)}\})=-L/4+\sum\limits_{n,\gamma}\frac{2n}{
(x_\gamma^{(n)})^2+n^2}\\
& P(\{x_\gamma^{(n)}\})=\sum\limits_{n,\gamma}\frac{2\pi 
I_\gamma^{(n)}}{L}
\end{align}
%
One can show that the Bethe-Takahashi quantum numbers $I_\gamma^{(n)}$ obey 
the constraint 
%
\begin{equation}
\label{bt-bounds}
|I_\gamma^{(n)}|\le\frac{1}{2}(L-1-\sum\limits_{m=1}^Mt_{mn}\alpha_m),
\end{equation}
%
where $t_{mn}\equiv 2 min(m,n)-\delta_{mn}$. 

\subsection{The Hilbert space Monte Carlo approach}


For a given number of particles $M$ the total number of eigenstates with 
that number of particles is clearly given as $C_M^L-C_{M-1}^L$, with 
$C_y^x\equiv x!/(y!(x-y)!)$ the binomial coefficient. Thus the probability 
of an eigenstate of the $XXX$ chain with $M$ particles is given as 
$(C_M^L-C_{M-1}^L)/C_{L/2}^L$. 

Given a fixed particle number $M$ the total number of eigenstates $D(\{\alpha_n\})$ 
corresponding to a string configuration $\{\alpha_n\}$ is given as 
%
\begin{equation}
D(\{\alpha_n\})=\prod_{i=1}^MC_{\alpha_i}^{L-\sum_{j=1}^M t_{ij}\alpha_j}.
\end{equation}
%
Given the fixed particle number $M$  and the string configuration $\{\alpha_j\}$, 
the $GGE$ probability $P_{GGE}$ of a generic eigenstate $\mu$ corresponding to a given 
choice of quantum numbers satisfying~\eqref{bt-bounds} is given as 
%
\begin{equation}
P_{GGE}(\mu)=\frac{1}{Z}(L-2M+1)e^{-\sum_i\lambda_j {\mathcal I}_j}.
\end{equation}
%
Here the factor $L-2M+1$ corresponds to the $SU(2)$ degeneracy. Notice that 
this assumes that all the conserved charges ${\mathcal I}_j$ are $SU(2)$ 
scalars. 

Given two eigenstates $\mu$ and $\nu$ corresponding to eigenvalues of the conserved 
charges $\{M,{\mathcal I_2},{\mathcal I}_3,\dots,{\mathcal I}_N\}$ and 
$\{M',{\mathcal I}'_2,{\mathcal I}'_3,\dots,{\mathcal I}'_N\}$ the transition 
probability is given as 
%
\begin{equation}
\frac{P_{GGE}(\nu)}{P_{GGE}(\mu)}=\frac{L-2M'+1}{L-2M+1}e^{-\sum_j
\lambda_j({\mathcal I}'_j-{\mathcal I}_j)}.
\end{equation}
%
This gives the Metropolis rule to be used in the Monte Carlo update as 
%
\begin{displaymath}
T(\mu\rightarrow\nu)=\left\{
\begin{array}{cc}
\frac{L-2M'+1}{L-2M+1}e^{-\sum_j\lambda_j({\mathcal I}'_j-{\mathcal I}_j)}.  &  \textrm{if}\,\frac{P(\nu)}{P(\nu)}<1\\
1									     &  \textrm{otherwise}.
\end{array}
\right.
\end{displaymath}
%



%########################################################################
\section{The conserved charges and their fluctuations}



%########################################################################
\section{The string root densities}


%%%%%%%%%%%%%%%%%%%%%%%%%%%%%%%%%e
\begin{figure*}[t]
\includegraphics*[width=0.93\linewidth]{./draft_figs/fig1}
\caption{The Generalized Gibbs Ensemble (GGE) for the Heisenberg spin chain with 
 $L=16$ sites: numerical results obtained using the Hilbert space Monte 
 Carlo sampling. Only the first three conserved charges ${\mathcal I}_n$ ($n=1,2,3$), 
 with associated Lagrange multipliers $\lambda_n$, are included in the GGE. Here 
 ${\mathcal I}_2$ is the Hamiltonian and $\lambda_2\equiv\beta$ the inverse 
 temperature. In all the panels different symbols correspond to different  values 
 of $\lambda_3,\lambda_4$. The circles correspond to the Gibbs ensemble, i.e., 
 $\lambda_3=\lambda_4=0$. (a) The GGE average $\langle {\mathcal I}_2/L\rangle$ 
 plotted as a function of $\beta$. (b) Variance of the GGE fluctuations $\sigma^2(
 {\mathcal I}_2/L)\equiv \langle ({\mathcal I}_2/L)^2\rangle-\langle {\mathcal I}_2/L
 \rangle^2$ as a function of $\beta$. (c)(d) and (e)(f): Same as in (a)(b) for 
 ${\mathcal I}_3$ and ${\mathcal I}_4$, respectively. In all panels the dash-dotted 
 lines  are the analytical results obtained using the Generalized Thermodynamic Bethe 
 Ansatz (GTBA). (g) The GGE expectation value of the total magnetization 
 $\langle S_z\rangle$. Notice that $\langle S_z\rangle=0$ due to the $SU(2)$ 
 invariance of the conserved charges. (h) $\chi/\beta$ plotted versus $\beta$, with 
 $\chi$ being the magnetic susceptibility per site. 
}
\label{fig1}
\end{figure*}
%%%%%%%%%%%%%%%%%%%%%%%%%%%%%%%%%%

%%%%%%%%%%%%%%%%%%%%%%%%%%%%%%%%%e
\begin{figure*}[t]
\includegraphics*[width=0.99\linewidth]{./draft_figs/fig2}
\caption{The rapidity densities $\rho_n(x)$ (for $n=1,2,3$) for the infinite temperature 
 Gibbs (panels (a)-(c)) and the GGE equilibrium states (panels (d)-(f)): Numerical  
 results for the Heisenberg spin chain obtained using the Hilbert space Monte Carlo 
 sampling. Here the GGE is constructed including only ${\mathcal I}_2$ and ${\mathcal I}_4$ 
 with fixed Lagrange multipliers $\lambda_2=0$ and $\lambda_4=1$. In all the panels the 
 data are the histograms of the $n$-strings rapidities sampled in the Monte Carlo.
 The width of the histogram bins is $\Delta x=2/L$. In each panel different histograms 
 correspond to different chain sizes $L$. All the histograms are divided by $10^3$ for 
 convenience. In (b) the arrow is to highlight the finite-size effects. In panels (a)-(c) 
 the lines are the Thermodynamic Bethe Ansatz (TBA) results. (g) Finite-temperature 
 effects: Monte Carlo data for $\rho^{\textrm{Gibbs}}_1$ for different values of the 
 inverse temperature $\beta$.
}
\label{fig2}
\end{figure*}
%%%%%%%%%%%%%%%%%%%%%%%%%%%%%%%%%%

%%%%%%%%%%%%%%%%%%%%%%%%%%%%%%%%%e
\begin{figure}[t]
\includegraphics*[width=0.93\linewidth]{./draft_figs/fig3}
\caption{Finite-size scaling of the GGE averages in the Heisenberg chain: Numerical results 
 obtained from the Hilbert space Monte Carlo sampling. Here the GGE is constructed including 
 ${\mathcal I}_2, {\mathcal I}_3, {\mathcal I}_4$, with Lagrange multipliers $\lambda_2=\beta,
 \lambda_3=\lambda_4=1$. (a) $\langle {\mathcal I}_2/L\rangle$ plotted versus the chain size 
 $L$ for several values of $\beta$. The dash-dotted lines are exponential fits. (b) Same as 
 in (a) for ${\mathcal I}_4$.
}
\label{finite-size}
\end{figure}
%%%%%%%%%%%%%%%%%%%%%%%%%%%%%%%%%%

For infinite temperature the densities $\rho_n$ are given as 
%
\begin{equation}
\rho_n(x)=\sqrt{\frac{2}{\pi}}\frac{a_n}{n^2(n+2)^2+(2(n+1)^2+2)x^2+x^4}
\end{equation}
%
where the sequence $a_n$ is given as 
%
\begin{equation}
a_n=\frac{2(n+1)^2}{(n+1)^2+1}
\end{equation}
%


%##########################################################################
\begin{thebibliography}{99}


\end{thebibliography}

\end{document}



