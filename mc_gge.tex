\documentclass[twocolumn,superscriptaddress,prb,10pt]{revtex4-1}
%\usepackage{verbatim}
\usepackage{amsmath,amssymb}
\usepackage{graphicx}
\usepackage{color}
\usepackage[colorlinks,bookmarks=false,citecolor=blue,linkcolor=red,urlcolor=blue]{hyperref}
\usepackage{times}





%\usepackage[dvips]{graphics}



%%%%%%%%%%%%%   useful shortcuts %%%%%%%%%%%%%%%%%%%%%%%%%%%%%%%%%

\def \h{\hbar}   %  \h won't be used for any greek letter
\def \refe{\eqref}
\def \trm{\textrm}
\def \f{\frac}
\def \hf{\tfrac{1}{2}}    \def \HF{\dfrac{1}{2}}
\def \u{\uparrow}
\def \d{\downarrow}

\def \ord{\mathcal{O}}
\newcommand{\ra}{\rightarrow}   \newcommand{\lra}{\longrightarrow}  

\def\lba{\left(}    \def\rba{\right)}
\def\lbc{\left[}    \def\rbc{\right]}
\def\lbb{\left\{}    \def\rbb{\right\}}

\def\tr{\textrm{Tr}}
\def\refe{\eqref}

\newcommand{\bra}[1]{\langle\left.{#1}\right|}
\newcommand{\ket}[1]{\left|{#1}\right.\rangle}
\newcommand{\xpct}[1]{\langle{#1}\rangle}    % expectatn value

%\DeclareMathOperator{\tr}{tr}

%%%%%%%%%%%%%%%%%%%%%%%%%%%%%%%%%%%%%%%%%%%%%%%%%%%%%%%

\newcommand{\vp}{{\bf p}}  % usual vector quantities
\newcommand{\vq}{{\bf q}}  % double bracketing not required with \vec
\newcommand{\vk}{{\bf k}}  % but required with \bf

\renewcommand{\vr}{{\bf r}} 
\newcommand{\vx}{{\bf x}}

\newcommand{\hc}{\hat{c}}  \newcommand{\hcd}{\hat{c}^\dag} 
\newcommand{\hd}{\hat{d}}  \newcommand{\hdd}{\hat{d}^\dag} 






\begin{document}

\title{The Generalized Gibbs Ensenble in the Heisenberg spin chain: A Hilbert space Monte Carlo 
approach} 

\author{Vincenzo Alba}
\affiliation{International School for Advanced Studies (SISSA),
Via Bonomea 265, 34136, Trieste, Italy, 
INFN, Sezione di Trieste}

\author{Maurizio Fagotti}
\affiliation{
D\'epartement de Physique, Ecole normale superieure, CNRS, 24 rue Lhomond, 75005
Paris, France}

\date{\today}




\begin{abstract} 


\end{abstract}

% \pacs{73.43.Cd, 71.10.Pm  {\tt check!}}

\maketitle


%########################################################################
\section{Introduction}

%%%%%%%%%%%%%%%%%%%%%%%%%%%%%%%%%e
\begin{figure*}[t]
\includegraphics*[width=0.93\linewidth]{./draft_figs/fig1}
\caption{The Generalized Gibbs Ensenble (GGE) for the finite-size Heisenberg spin 
 chain with $L=16$ sites. The GGE is constructed including the conserved charges  
 $I_2, I_3,I_4$. The corresponding Lagrange multipliers are denoted  as $\lambda_2,
 \lambda_3,\lambda_4$. Here $I_2$ is the Hamiltonian and $\lambda_2\equiv\beta$ the 
 inverse temperature. (a) The GGE average $\langle I_2/L\rangle$ of $I_2/L$ plotted 
 as a function of $\beta$. The data are obtained using the Hilbert space Monte Carlo 
 approach described in the manuscript. The different symbols correspond to GGEs with 
 different fixed  values of $\lambda_3$ and $\lambda_4$. The circles correspond to the 
 Gibbs ensemble. (b) The fluctuations $\sigma^2(I_2)/L\equiv \langle (I_2/L)^2\rangle-
 \langle I_2/L\rangle^2$ as function of $0\le\beta\le 1.5$. (c)(d) and (e)(f): Same 
 as in (a)(b) for $I_3$ and $I_4$, respectively. In all panels the dash-dotted lines  
 are the analytical results obtained using the Generalized Thermodynamic Bethe 
 Ansatz (GTBA) approach.
}
\label{fig1}
\end{figure*}
%%%%%%%%%%%%%%%%%%%%%%%%%%%%%%%%%%

%%%%%%%%%%%%%%%%%%%%%%%%%%%%%%%%%e
\begin{figure}[t]
\includegraphics*[width=0.93\linewidth]{./draft_figs/fig2}
\caption{The Generalized Gibbs Ensenble (GGE) in the Hisenberg spin chain of 
 length $L=16$. The GGE is obtained including the first three non-trivial 
 conserved quantity $I_2, I_3,I_4$. Here $I_2$ is the Hamiltonian. The corresponding 
 Lagrange multipliers are denoted as $\lambda_2,\lambda_3,\lambda_4$, with $\lambda_2$ 
 being the inverse temperature $\lambda_2=\beta$. In all the panels circles, squares, 
 and rhombi correspond to the the situations with $\lambda_3=\lambda_4=0$ 
 (i.e., the Gibbs ensenble), $\lambda_3=1,\lambda_4=0$, and $\lambda_3=\lambda_4=1$.
 The GGE expectation value for the fluctuations of the total magnetization $M$, plotted 
 as a function of $\lambda_2$.
}
\label{fig2}
\end{figure}
%%%%%%%%%%%%%%%%%%%%%%%%%%%%%%%%%%

%%%%%%%%%%%%%%%%%%%%%%%%%%%%%%%%%e
\begin{figure}[t]
\includegraphics*[width=0.93\linewidth]{./draft_figs/finite_size}
\caption{The Generalized Gibbs Ensenble (GGE) in the Hisenberg spin chain of 
 length $L=16$. The GGE is obtained including the first three non-trivial 
 conserved quantity $I_2, I_3,I_4$. Here $I_2$ is the Hamiltonian. The corresponding 
 Lagrange multipliers are denoted as $\lambda_2,\lambda_3,\lambda_4$, with $\lambda_2$ 
 being the inverse temperature $\lambda_2=\beta$. In all the panels circles, squares, 
 and rhombi correspond to the the situations with $\lambda_3=\lambda_4=0$ 
 (i.e., the Gibbs ensenble), $\lambda_3=1,\lambda_4=0$, and $\lambda_3=\lambda_4=1$.
 The GGE expectation value for the fluctuations of the total magnetization $M$, plotted 
 as a function of $\lambda_2$.
}
\label{finite-size}
\end{figure}
%%%%%%%%%%%%%%%%%%%%%%%%%%%%%%%%%%


%##########################################################################
\begin{thebibliography}{99}


\end{thebibliography}

\end{document}



